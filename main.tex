\documentclass[dvipsnames, svgnames, leqno, a4paper, 12pt]{article}
\usepackage{preconfig}

\title{El Problema d'interpolació de Pick--Nevanlinna}
\date{}
\author{Daniel Benages}
\begin{document}
\tikzset{font=\scriptsize}
\shorthandoff{"}
    \begin{titlepage}
      \maketitle

    \end{titlepage}

    \begin{abstract}
        L'anàlisi complexa sol destacar per l'elegància i finor dels seus teoremes i demostracions, des del Teorema de representació en sèries de potències fins al Teorema global de Cauchy. Aquest treball, però, busca resoldre un problema molt més concret. Allunyem-nos de Teoremes omnipotents, de resultats que ressonen per tot el món complex. Centrem-nos en un petit disc de radi 1. Mirem dos conjunts finits de punts. Hi pot haver una funció holomorfa que els aparelli? 
        
        El camí que recorre aquest treball termina amb el resultat que va donar Georg Pick l'any 1917. Veurem com l'anàlisi complexa, amb la seva elegància habitual, redueix el problema a l'estudi del signe d'una forma quadràtica. Això serà gràcies al Lema de Schwarz-Pick, una generalització natural del conegut Lema de Schwarz.


    \end{abstract}
\newpage
    \section{Introducció}
        L'objectiu d'aquest treball és donar un resultat sobre el següent problema d'interpolació:
        \begin{problem}\label{problema}
            Siguin $z_1,\dots,z_n$ punts diferents dins el disc unitat $\D$. Quines condicions han de complir $w_1,\dots,w_n$ per tal de que existeixi una funció $f$ amb\begin{equation}
                f(z_j)=w_j,\quad j=1,\dots,n
            \end{equation}
            i $f\colon\D\to\D$ holomorfa?
        \end{problem}
        Per a això, començarem amb un breu estudi de les homografies. Aquestes ens permetran endinsar-nos en els automorfismes de \(\D\) i demostrar dos resultats de Schwarz i de Pick.

        Posteriorment parlarem de l'última peça clau: els productes de Blaschke finits. Finalment, demostrarem el teorema de Pick:
        \begin{theorem*}[Pick]
            Existeix solució per al problema \normalfont{\ref{problema}} si i només si la forma quadràtica \begin{displaymath}
                Q_n(t_1,\dots,t_n)=\sum_{j,k=1}^n=\frac{1-w_j\overline{w_k}}{1-z_j\overline{z_k}}t_j\overline{t_k}
            \end{displaymath}
            és semidefinida positiva. En aquest cas, existeix un producte de Blaschke finit de grau com a molt $n$ que resol \normalfont{\ref{problema}}.
        \end{theorem*}

        Aquesta tasca, però, és inabordable sense fer algunes concessions. La més important és que donarem per fetes la majoria de nocions que s'obtindrien en un curs elemental d'anàlisi complexa. Si més no, fins al teorema del mòdul màxim (tot i que l'enunciarem per refrescar la memòria).



    \section{Homografies}
        Sigui  \(\Omega\) un domini de \(\mathbb{C}\) i \(f: \Omega\to\Omega\) una funció holomorfa, diem que \(f\) és \textbf{conforme} si \(f'(z) \neq 0\,\  \forall z\in\Omega\).
        
        Diem \textbf{esfera de Riemann} a la compactificació del pla complex per un punt. La manera usual de pensar en aquest espai topològic és considerar una esfera on el pol nord és \(\infty\). La denotem per \(S^2\).

        \begin{figure}[H]
            \centering

            \begin{tikzpicture}
                \coordinate (aux) at (0, -2);
                \coordinate (aux2) at (0, 2);
                \shade[ball color = gray!60, opacity = 0.6] (0,0) circle (2cm);
                \draw (0,0) circle (2cm);
                \draw (-2,0) arc (180:360:2 and 0.6) node[pos=.5] (A) {};
                \draw ($(A) + (0, 1pt)$) -- ($(A) + (0,-1pt)$) node [pos=.5, below]{$-i$};
                \draw[dashed] (2,0) arc (0:180:2 and 0.6) node[pos=.5] (B) {};
                \draw ($(B) + (0, 1.1pt)$) -- ($(B) + (0,-1.1pt)$) node [pos=.5, above]{$i$};
                \draw[cm={cos(90) ,-sin(90) ,sin(90) ,cos(90) ,(0 cm,0 cm)}] (2,0) arc (0:180:2 and 0.6);
                \draw[dashed,cm={cos(90) ,-sin(90) ,sin(90) ,cos(90) ,(0 cm,0 cm)} ] (-2,0) arc (180:360:2 and 0.6);
                \fill[fill=black] (0,0) circle (1pt);
                \draw (2 cm,1pt) -- (2 cm,-1pt) node[anchor=west] {1};
                \draw (-2 ,1pt) -- (-2,-1pt) node[anchor=east] {-1};
                \draw ($(aux) + (0,1pt)$)-- ($(aux) + (0,-1pt)$) node[anchor=north] {0};
                \draw ($(aux2) + (0,1pt)$)-- ($(aux2) + (0,-1pt)$) node[anchor=south] {$\infty$};
              \end{tikzpicture}

            \caption{L'esfera de Riemann.}
        \end{figure}
        No ens cal preocupar-nos per les propietats topològiques de \(S^2\). Per al que a nosaltres ens ocupa, l'esfera de Riemann es comporta com \(\mathbb{C}\), però ens permet tractar \(\infty\) estalviant-nos límits. És fàcil veure que \(1/0 = \infty\) i que \(1/\infty = 0\).

        De fet, aquest és el motiu principal per introduir el concepte de \(S^2\), ja que serveix de "cèrcol" per no deixar escapar els punts que habitualment hauríem de considerar pols de funcions altrament holomorfes.

        Siguin \(a,b,c,d\in S^2\). Una \textbf{homografia} és una aplicació de la forma
        \begin{displaymath}
            T(z)=\frac{az+b}{cz+d},\quad ad-bc\neq0.
        \end{displaymath}
        Observem que si $z\neq -d/c$, les homografies són holomorfes a $\C$ i $ad-bc\neq0$ ens dona conformitat.
        
        \begin{proposition}
            Les homografies són invertibles i la seva inversa és una homografia.
        \end{proposition}
        
        \begin{proof}
            Si \begin{math}
                 w  = \frac{az+b}{cz+d} 
            \end{math}, podem aïllar $z$ i obtenim \begin{math}
                z = \frac{-d  w +b}{c w -a}
            \end{math}.
        \end{proof}

        \begin{corollary}
            Les homografies són bijeccions holomorfes de $S^2$.
        \end{corollary}

        \begin{lemma}
            Sigui $f$  una funció holomorfa tret d'un nombre finit de punts. Si $f$ té un pol d'ordre igual o superior a 2, $f$ no pot ser injectiva.  
        \end{lemma}

        \begin{proof}
            Considerem la funció $1/f$. $f$ és injectiva si i només si $1/f$ ho és. Per tant demostrar la no injectivitat de $1/f$ és suficient.
            
            Sigui $a$ un pol d'ordre $m>1$ de $f$. Llavors $a$ és un zero d'ordre $m$ de $1/f$, per tant 
            \begin{displaymath}
                \frac{1}{f}=(z-a)^mh(z),\quad  h(a)\neq0
            \end{displaymath} 
            per a una certa funció holomorfa $h$.

            Considerem $ g(z)=(z-a)h(z)^{1/m}$. Aquesta funció és holomorfa i ben definida sobre l'arrel principal $m$-èssima. La seva derivada és 
            \begin{displaymath}
                g'(z)=\frac{1}{m}(z-a)h(z)^{\frac{1}{m}-1}h'(z)+h(z)^{\frac{1}{m}}
            \end{displaymath} 
            per tant $g'(a)\neq0$.
            Tenim, doncs, un entorn de $a$ on $g$ és una funció holomorfa i invertible que envia $a$ al zero. Amb el canvi de coordenades $ w  = g(z)$, veiem que localment \begin{displaymath}
                 w ^m = \frac{1}{f(g^{-1}(z))}
            \end{displaymath} 
            per tant en un entorn de $a$ la funció $1/f$ es comporta com la funció $z\to z^m$, que per $m>1$ no és injectiva.
        \end{proof}

        \begin{theorem}
            Si, tret d'un nombre finit de punts de $S^2$, tenim una bijecció conforme entre $S^2$ i una regió del propi $S^2$, aquesta bijecció és una homografia.
        \end{theorem}

        \begin{proof}
            Sigui $f$ aquesta bijecció i $q_1,\dots,q_n$ els punts exclòsos. Com $f$ és conforme, $f$ és holomorfa tret de en $q_i$. Aquests punts només poden ser pols, singularitats essencials o evitables. 
            
            La possibilitat de que siguin essencials queda descartada, ja que en tot entorn d'una discontinuïtat d'aquest tipus la imatge és tot el pla complex, per tant $f$ no podria ser bijectiva. Com les discontinuïtats de $f$ són o evitables o pols, podem assegurar que $f$ es una funció racional. 
            
            A més, per la injectivitat a $S^2$, només un dels punts pot ser un pol, el qual pel lema anterior seria d'ordre 1. Si aquest es troba en un punt finit $q_k$, \begin{displaymath}
                f(z)=\frac{A_1}{z-q_k}+A_0=\frac{A_0z+A_1-A_0q_k}{z-q_k}\,\text{, } A_1\neq0.
            \end{displaymath}
            Si $q_k=\infty$, 
            \begin{displaymath}
                f(z)=A_1z+A_0\, \text{, } A_1\neq0.
            \end{displaymath}
            Sigui com sigui, la funció és una homografia.
        \end{proof}

        \begin{proposition}\label{prop:comp_homo}
            La composició finita d'homografies és equivalent a una sola homografia.
        \end{proposition}

        \begin{proof}
            És suficient demostrar-ho per la composició de dues homografies. Siguin 
            \begin{displaymath}
                T(z)=\frac{az+b}{cz+d},
            \end{displaymath}
            \begin{displaymath}
                S(z)=\frac{a'z+b'}{c'z+d'}.
            \end{displaymath}
            Llavors
            \begin{displaymath}
                (T\circ S)(z)=\frac{a\frac{a'z+b'}{c'z+d'}+b}{c\frac{a'z+b'}{c'z+d'}+d}=\frac{aa'z+ab'+bc'z+bd'}{ca'z+cb'+dc'z+dd'}=\frac{\left( aa'+bc' \right)z+(ab'+bd')}{\left( ca'+dc' \right)z+(cb'+dd')},
            \end{displaymath}
            \normalsize
            que clarament és una homografia.
        \end{proof}

        Degut a això, podem descompondre totes les homografies en combinacions de tres classes fonamentals:
        
        \begin{theorem}
            Tota homografia és composició finita de translacions, rotacions, homotècies i inversions.
        \end{theorem}
        
        \begin{proof}
            Totes aquestes transformacions són clarament homografies. Veiem que podem crear una cadena de composicions que ens porti a qualsevol homografia:
            
            Si $c=0$, simplement $z\to az\to az+b\to \frac{az+b}{d}$. Si $c\neq0$, llavors 
            \begin{align*}
                &z\to cz\to cz+d\to \frac{1}{cz+d}\to \frac{\frac{bc-ad}{c}}{cz+d}\\
                &\to \frac{\frac{bc-ad}{c}}{cz+d}+\frac{a}{c}=\frac{bc-ad}{c(cz+d)}+\frac{a}{c}=\frac{bc-ad+azc+ad}{c(cz+d)}=\frac{az+b}{cz+d}.
            \end{align*}
            En qualsevol dels casos, arribem a l'homografia.
        \end{proof}

        \noindent Definim la \textbf{raó doble} entre quatre nombres complexos $z_1,z_2,z_3,z_4$ per 
        \begin{displaymath}
                (z_1,z_2,z_3,z_4) = \frac{(z_1-z_3)(z_2-z_4)}{(z_2-z_3)(z_1-z_4)}.
        \end{displaymath}
            %
        Ens serà útil considerar que algun d'aquests punts sigui $\infty$. En aquest cas, ometrem els termes que l'incloguin. Per exemple, si $z_1 = \infty$, llavors 
        \begin{displaymath}
                (\infty, z_2,z_3,z_4) = \frac{z_2-z_4}{z_1-z_3}.
        \end{displaymath}

        \noindent Donats $z_1,z_2,z_3\in\mathbb{C}$, la funció donada per la raó doble 
        
        \begin{equation}\label{eq:rao_doble}
            (z,z_1,z_2,z_3) = \frac{(z-z_2)(z_1-z_3)}{(z-z_3)(z_1-z_2)}=\frac{(z_1-z_3)z-z_2(z_1-z_3)}{(z_1-z_2)z-z_3(z_1-z_2)}
        \end{equation}
        %
        és una homografia que envia $z_1\to1$, $z_2\to0$ i $z_3\to\infty$.

        A més, si algun $z_i$ és $\infty$, utilitzant la raó doble que pertoca es segueix complint aquesta afirmació.

    \begin{proposition}
        Les homografies conserven la raó doble.
    \end{proposition}

    \begin{proof}
        Siguin $z_1,z_2,z_3,z_4\in \C$ (finits); $z'_1,z'_2,z'_3,z'_4$ la seva imatge per una homografia que no envii cap a $\infty$. Tenim 
        \footnotesize
        \begin{align*}
            z'_1-z'_3 &=\frac{az_1+b}{cz_1+d}-\frac{az_3+b}{cz_3+d}\\
            &= \frac{acz_1z_3+adz_1+bcz_3+bd-acz_1z_3-adz_3-bcz_1-bd}{(cz_1+d)(cz_3+d)}
            = \frac{(ad-bc)(z_1-z_3)}{(cz_1+d)(cz_3+d)}.
        \end{align*}
        \normalsize
        Ídem per a $z'_1-z'_4, z'_2-z'_4, z'_2-z'_3$, per tant un cop simplifiquem tenim 
        \begin{displaymath}
            \frac{(z'_1-z'_3)(z'_2-z'_4)}{(z'_1-z'_4)(z'_2-z'_3)}=\frac{\frac{(ad-bc)^2(z_1-z_3)(z_2-z_4)}{(cz_1+d)(cz_2+d)(cz_3+d)(cz_4+d)}}{\frac{(ad-bc)^2(z_1-z_4)(z_2-z_3))}{(cz_1+d)(cz_2+d)(cz_3+d)(cz_4+d)}} = \frac{(z_1-z_3)(z_3-z_4)}{(z_1-z_4)(z_2-z_3)}.
        \end{displaymath}
        Per últim, suposem un dels punts és $\infty$. Per exemple, sigui $z'_1=\infty$, tenim que $d=-cz_1$. La raó doble de les imatges és 
        \begin{align*}
            (\infty,z'_2,z'_3,z'_4) &=\frac{(z'_2-z'_4)}{(z'_2-z'_3)}=\frac{(z_2-z_4)(cz_3+d)}{(z_2-z_3)(cz_4+d)}=\frac{(z_2-z_4)(cz_3-cz_1)}{(z_2-z_3)(cz_4-cz_1)}\\
            &=\frac{(z_1-z_3)(z_2-z_4)}{(z_1-z_4)(z_2-z_3)},
        \end{align*}
        per tant es conserva.
    \end{proof}

    \begin{theorem}
        L'única homografia que deixa fixos més de dos punts és la identitat.
    \end{theorem}

    \begin{proof}
        Donada l'homografia $\frac{az+b}{cz+d}$, els seus punts fixos compleixen 
        \begin{displaymath}
            \frac{az+b}{cz+d}=z\implies cz^2+(d-a)z+b=0.
        \end{displaymath}
        Si $c=0$, només hi ha un punt fix $z=\frac{b}{d-a}$. Si $c\neq0$, tenim una equació de segon grau i l'única forma que es compleixi per a tot $z$ és que $c=b=0$ i $a=d$, per tant 
        \begin{displaymath}
            \frac{az+b}{cz+d}=\frac{az}{a}=z\implies \text{és la aplicació identitat.}
        \end{displaymath}
    \end{proof}

    Això ens porta al teorema de caracterització de les homografies.

    \begin{theorem}
        Tota homografia està definida per les imatges de tres punts diferents.
        Dit d'una altra manera, donats $z_1,z_2,z_3,z_4,z'_1,z'_2,z'_3,z'_4\in\C$, existeix una única homografia $T$ tal que $T(z_1)=z'_1$, $T(z_2)=z'_2$, $T(z_3)=z'_3$, $T(z_4)=z'_4$.
    \end{theorem}

    \begin{proof}
        Demostrem primer la unicitat. Siguin $S$, $T$ dues homografies tals que $S(z_i)=T(z_i)=z'_i$. Llavors, per Proposició \ref{prop:comp_homo}, $S^{-1}T$ és una homografia que compleix 
        \begin{displaymath}
            S^{-1}T(z_i)=S^{-1}(z'_i)=z_i\, i=1,\dots4.
        \end{displaymath}
        Com té quatre punts fixos, $S^{-1}T=id\implies T=S$.

        Per veure l'existència, recordem que per (\ref{eq:rao_doble}), la rao doble $(z,z_1,z_2,z_3)$ ens envia $z_1\to1,\, z_2\to0,\, z_3\to\infty$. 

        Siguin $S=(z,z'_1,z'_2,z'_3)$ i $T=(z,z_1,z_2,z_3)$
        Llavors, la composició 
        \begin{displaymath}
            S^{-1}\circ T
        \end{displaymath}
        és la homografia que busquem.
    \end{proof}




\section[Automorfismes al Disc Unitat]{Automorfismes a $\D$}
    Com ja hem comentat prèviament, centrarem la nostra atenció exclusivament a $\D$. Per tant, els teoremes i proposicions que enunciarem a partir d'ara seran sempre en aquesta regió. Recordem que definim com Disc Unitat
    
    \begin{equation*}
            \D:=\{\, z\in\C\mid |z|<1\, \}.
    \end{equation*}

    \noindent Observem que $\D$ és un domini acotat de $\C$.

    Enunciem ara el Teorema del Mòdul Màxim que, tot i que no el demostrarem, ens serà indispensable per als propers resultats. 
    
    \begin{theorem}[Teorema del Mòdul Màxim]\label{th:TMM}
        Sigui $K$ la clausura d'una regió acotada $\Omega$. Si $f$ és continua en $K$ i holomorfa en $\Omega$, llavors \begin{equation}
            \abs{f(z)}\leq\norm{f}_{\partial\Omega},\;\forall z\in\Omega.
        \end{equation}
        Si es dona la igualtat per a algun $z\in\Omega$, llavors $f$ és constant.
    \end{theorem}
    
    Anomenarem $\mathbbb{D}$\textbf{-holomorfisme} o funció $\mathbbb{D}$\textbf{-holomorfa} a totes aquelles funcions $\D\to\D$ holomorfes.
    
    Enunciem ara el Teorema de caracterització dels automorfismes holomorfs de $\D$:
    
    \begin{theorem}
        Una funció $T$ $\D$-holomorfa és bijectiva si i només si
        
        \begin{equation}
            T(z)=\lambda\frac{a-z}{1-\overline{a}z},\, \text{amb }a\in\D\; \text{ i }\; \abs{\lambda}=1.
        \end{equation}

    \end{theorem}

    Dit d'una altra manera, els automorfismes del disc són una classe molt concreta d'homografies.

    \begin{proof}
        Pel que ja hem vist, sabem que $T(z)=\lambda\frac{a-z}{1-\overline{a}z}$ és automorfisme holomorf de $\C$. Només cal veure que és bijectiu a $\D$. 

        Sigui $z\in\D$. Tenim
        \begin{align*}
            \frac{\abs{a-z}}{\abs{1-\overline{a}z}}<1 &\iff \abs{a-z}^2<\abs{1-\overline{a}z}^2\\
            &\iff\abs{a}^2+\abs{z}^2-2\Re{a\overline{z}}<1+\abs{\overline{a}z}^2-2\Re{\overline{\overline{a}z}}\iff\\
            &\iff\abs{a}^2\abs{z}^2-\abs{a}^2\abs{z}^2-1<0\iff\left( \abs{z}^2-1 \right)\left( 1-\abs{a}^2 \right)<0\\
            &\iff\abs{a}<1.
        \end{align*}
        Per tant $\D\xrightarrow{T} \D$ és automorfisme holomorf.

        Vegem ara la implicació contraria. Sigui $T$ un $\D$-holomorfisme bijectiu. 
        Suposem primer que $T(0)=0$. Com $T$ és holomorfa, és una serie de potencies i si deixa el 0 fix, $\frac{T(z)}{z}$ és holomorfa en $\D$. Considerem $\abs{\frac{T(z)}{z}}$ en un disc $U$ de radi $r<1$ centrat a l'origen. Pel Teorema del Mòdul Màxim \ref{th:TMM}, $T(z)/z$ és màxim en $\abs{z}=r$. Llavors en $U$ tenim 
        \begin{displaymath}
            \abs{\frac{T(z)}{z}}<\frac{1}{r}\xrightarrow{r\to1}\abs{\frac{T(z)}{z}}\leq1,\, \text{en }\D.
        \end{displaymath}
        Sigui $S$ la inversa de $T$. Pel mateix raonament tenim $\abs{\frac{S(z)}{z}}<1$ en $\D$. Així doncs,
        \begin{displaymath}
            \frac{S(z)}{z}=\frac{z}{T(z)}\implies \abs{\frac{T(z)}{z}}=1,\; \forall z\in\D.
        \end{displaymath}
        i com el valor absolut és constant, també ho és la funció (pel fet de ser holomorfa). Llavors 
        
        \begin{equation}
            \frac{T(z)}{z}=e^{i\alpha}\implies T(z)=e^{i\alpha}z
        \end{equation}
        %
        per una $\alpha\in\mathbb{R}$ constant. Per tant, un automorfisme bijectiu de $\D$ que fixa l'origen és una rotació. 

        Eliminem la restricció de que l'origen sigui fix. Sigui $T(z)=a\neq0$. Considerem l'homografia 
        \begin{displaymath}
            R=\lambda\frac{a-z}{1-\overline{a}z}
        \end{displaymath}
        amb $\lambda=e^{i\beta}$, $\beta\in\mathbb{R}$. La composició $R^{-1}T$ és un automorfisme holomorf de $\D$ que deixa fix l'origen. Per tant és una rotació, diguem-li $e^{i\alpha}$. 

        Llavors 
        \begin{displaymath}
            R^{-1}T=e^{i\alpha}\implies T=Re^{i\alpha}\implies T(z)=e^{i(\alpha+\beta)}\frac{a-z}{1-\overline{a}z}=\lambda'\frac{a-z}{1-\overline{a}z}.
        \end{displaymath}
    \end{proof}
    
    A partir d'ara, denotarem 
    
    \begin{equation}
        \varphi_a(z):=\frac{a-z}{1-\overline{a}z}.
    \end{equation}

    Vegem ara un clàssic de l'anàlisi complexa.

    \begin{theorem}[Lema de Schwarz]\label{th:sch}
        Sigui $f:\D\to\D$ una funció holomorfa amb $f(0)=0$.

        Aleshores
        \begin{enumerate}[(i)]
            \item $\abs{f(z)}\leq\abs{z}$ per a tot $z\in\D$.
            \item $\abs{f'(0)}\leq 1$.
        \end{enumerate}
        A més, si hi hagués igualtat en algun dels dos casos, llavors $f(z)=e^{i\alpha}z$ per a algun $\alpha\in\mathbb{R}$ i per a tot $z\in\D$.
    \end{theorem} 

    \begin{proof}
        Demostrarem primer \textit{(i)}.

        Com $f(0)=0$ i és holomorfa, tenim que $h(z)=f(z)/z$ és holomorfa en tot $\D$. Pel Teorema del mòdul màxim \ref{th:TMM}, 
        
        \begin{equation}
            \sup_{\abs{z}\leq r}\abs{h(z)}=\sup_{\abs{z}= r}\abs{h(z)}=\frac{1}{r}\sup_{\abs{z}\leq r}\abs{f(z)}
        \end{equation}
        %
        per a $0<r<1$. Com $\abs{f(z)}\leq 1$ per a tot $z\in\D$, fem tendir $r$ a 1 i tenim 
        
        \begin{equation}
            \sup_{z\in\D}\abs{h(z)}\leq 1\implies \abs{f(z)}\leq \abs{z}.
        \end{equation}
        
        Veiem també que si $\abs{f(z_0)}=\abs{z_0}$ per a algun $z_0\in\D\setminus\{0\}$, llavors $\abs{h(z_0)}=1$ i, pel teorema del mòdul màxim, $h$ és constant de mòdul 1. Així, $\exists\alpha\in\mathbb{R}$ tal que 
        \begin{displaymath}
            f(z)=zh(z)=ze^{i\alpha}, \; \forall z\in\D.
        \end{displaymath}
        
        Vegem ara \textit{(ii)}.

        Com $f(z)/z=h(z)$ i $f(0)=0$, llavors 
        
        \begin{equation}
            \abs{f'(0)}=\lim_{z\to0}\frac{\abs{f(z)}}{\abs{z}}=\lim_{z\to0}\abs{h(z)}=\abs{h(0)}\leq1.
        \end{equation} 
    
        A més, si $\abs{f'(0)}=\abs{h(0)}=1$, pel teorema del mòdul màxim un altre cop tenim $h=e^{i\alpha}$ per a un cert $\alpha\in\mathbb{R}$ i per tant $f(z)=e^{i\alpha}z$, $\forall z\in\D$.
    \end{proof}

    No tan famosa és una generalització d'aquest lema, on es relaxen les hipòtesis i no es suposa que $f(0)=0$. Aquest és el lema de Schwarz-Pick:
    
    \begin{theorem}[Lema de Schwarz-Pick]\label{lema:SP}
        Sigui $f\colon\D\to\D$ una funció holomorfa. Llavors
        \begin{enumerate}[(i)]
            \item \(\displaystyle \abs{\frac{f(z)-f( w )}{1-f(z)\overline{f( w )}}}\leq\frac{z- w }{1-z\overline{ w }},\text{ per a tot }z, w \in\D\).
            \item \(\displaystyle \frac{\abs{f'(z)}}{1-\abs{f(z)}^2}\leq\frac{1}{1-\abs{z}^2}\text{ per a tot }z\in\D\).
        \end{enumerate}
        La igualtat en tots dos casos es dona si $f$ és automorfisme de $\D$. Si la igualtat de \textit{(i)} es compleix per a uns $z, w$ amb $z\neq w $, o si es dona en \textit{(ii)} per a algun $z$, llavors $f$ és automorfisme de $\D$.
    \end{theorem} 

    \begin{proof}
        Sigui $g = \varphi_{f( w )}\circ f\circ \varphi_{- w }$. $g$ és un automorfisme de $\D$ holomorf. 
        \begin{sloppypar}Com \({\displaystyle g(z)=\varphi_{f( w )}\left( f\left( \frac{z+ w }{1+\overline{ w }z} \right) \right)}\), en particular tenim $g(0)=\varphi_{f( w )}\left( f( w ) \right)=0$. Pel lema de Schwarz \ref{th:sch}, $\abs{g(\xi)}\leq\abs{\xi}$ per a tot $\xi\in\D$ i 
        \begin{displaymath}
            \abs{g(\varphi_{f( w )}(z))}\leq\abs{\varphi_{ w }(z)}\implies\abs{\varphi_{f( w )}(f(z))}\leq\abs{\varphi_ w (z)}.
        \end{displaymath}
        Substituint cada terme per la seva definició 
        
        \begin{equation}
            \abs{\frac{f(z)-f( w )}{1-\overline{f( w )}f(z)}}\leq\abs{\frac{z- w }{1-\overline{ w }z}}.
        \end{equation}\end{sloppypar}
        
        Si tenim igualtat per uns $z\neq w $, llavors $\abs{g(\varphi_ w (z))}=\abs{\varphi_ w (z)}$ i pel lema de Schwarz, $g(z)=ze^{i\alpha}\implies g$ és un automorfisme de $\D\implies$ $f=\varphi_{-f( w )}\circ g\circ \varphi_ w $ també.

        Veiem ara \textit{(ii)}.

        \begin{displaymath}
            \frac{\abs{f'(z)}}{1-\abs{f(z)}^2}=\lim_{ w \to z}\left( \abs{\frac{f(z)-f( w )}{z- w }}\frac{1}{\abs{1-\overline{f( w )}f(z)}} \right)\leq\lim_{ w \to z}\frac{1}{1-\overline{ w }z}=\frac{1}{1-\abs{z}^2}.
        \end{displaymath}

        Si tenim igualtat a \textit{(ii)}, considerem $z= w $ i com \(\displaystyle \varphi_{a}'(z)=\frac{1-\abs{a}^2}{(1-\overline{a}z)^2}\), tenim 
        \begin{align*}
            \abs{g'(0)} &= \abs{\varphi_{f( w )}'(f\circ\varphi_{- w }'(0))}\abs{f'\left( \varphi_{- w }(0) \right)}\abs{\varphi_{- w }'(0)}\\
            &=\abs{\varphi_{f( w )}'\left( f( w ) \right)}\abs{f'( w )}\left( 1-\abs{ w }^2 \right)\\
            &=\frac{1-\abs{f( w )}^2}{\left( 1-\overline{f( w )}f( w ) \right)^2}\abs{f'( w )}\left( 1-\abs{ w }^2 \right)\\
            &=\frac{\abs{f'( w )}}{1-\abs{f( w )}^2}\left( 1-\abs{ w }^2 \right)=1
        \end{align*}
        i, pel Lema de Schwarz, $g$ és automorfisme i $f=\varphi_{-f( w )}\circ g\circ \varphi_ w $ també. 

        Finalment, si $f$ és automorfisme de $\D$, apliquem \textit{(i)} a $f$ i $f^{-1}$ i obtenim 
        
        \begin{equation}
            \abs{\frac{z- w }{1-z\overline{ w }}}=\abs{\frac{f^{-1}(f(z))-f^{-1}(f( w ))}{1-f^{-1}(f(z))\overline{f^{-1}(f( w ))}}}\leq\abs{\frac{f(z)-f( w )}{1-f(z)\overline{f( w )}}}\leq\abs{\frac{z- w }{1-z\overline{ w }}}
        \end{equation} 
        %
        per tant hi ha igualtat.
        De la mateixa manera, per a \textit{(ii)} tenim 
        
        \begin{equation}
            \frac{1}{1-\abs{z}^2}=\frac{\abs{(f^{-1}\circ f)'(z)}}{1-\abs{(f^{-1}\circ f)(z)}^2}\leq\frac{f'(z)}{1-\abs{f(z)}^2}\leq\frac{1}{1-\abs{z}^2}
        \end{equation} 
        
        aconseguint novament la igualtat.
    \end{proof}



\section{El Teorema de Pick}

\subsection{Productes de Blaschke finits}

L'última peça clau per al problema d'interpolació són els productes de Blaschke. No ens caldrà anar més enllà del cas finit.

Un \textbf{producte de Blaschke finit} és una funció de la forma 
\begin{displaymath}
    B(z)=e^{i\alpha}\prod_{j=1}^{n}\frac{z-z_j}{1-\overline{z_j}z},\;\ \abs{z_j}<1.
\end{displaymath}
Ens seran útils les següents propietats:

\begin{proposition}
    Si $B$ és un producte de Blaschke, és compleix:
    \begin{enumerate}[(i)]
        \item $B$ és continua en $\partial\D$.
        \item $\abs{B}=1$ en $\partial\D$.
        \item $B$ té un nombre finit de zeros en $\D$.
    \end{enumerate}
    Aquestes propietats, junt amb on són aquestes zeros, determinen $B$ tret d'una constant de mòdul 1.
\end{proposition}

\begin{proof}
    No demostrarem $(i)$, $(ii)$, $(iii)$; el que sí veurem és la part final.

    Sigui una funció $f$ que compleix les tres i $B$ un producte de Blaschke amb els mateixos zeros. Pel principi del mòdul màxim, $\abs{f/B}\leq1$ i $\abs{B/f}\leq1$ en $\D$. Per tant $f/B$ és constant.
\end{proof}

Els productes de Blaschke, a més de ser útils en la interpolació (com veurem pròximament),  juguen un paper important en l'aproximació de funcions del disc. 

\begin{theorem}[Carathéodory]
    Sigui $f$ un $\D$-holomorfisme. Existeix $\{B_k\}$ una successió de productes de Blaschke que convergeixen puntualment a $f$.
\end{theorem}

\begin{proof}
    Com $f$ és holomorfa, és una serie de potencies. Siguin $\{c_k\}$ els seus coeficients. Demostrarem el teorema per inducció, trobant un producte de Blaschke de grau com a molt $n$ tal que els $n$ primers coeficients coincideixen amb els de $f$, i.e., 
    \begin{displaymath}
        B_n=c_0+c_1z+\dots+c_{n-1}z^{n-1}+d_nz^n\dots
    \end{displaymath}
    Com $\abs{c_0}\leq1$, podem fixar $B_0=\frac{z+c_0}{1+\overline{c_0}z}$. Si $\abs{c_0}=1$, llavors $B_0=c_0$ és un producte de Blaschke de grau 0.

    Suposem que per a qualsevol $g$ $\D$-holomorfisme tenim construït el seu $B_{n-1}(z)$. Sigui 
    \begin{displaymath}
        g=\frac{1}{z}\frac{f-f(0)}{1-\overline{f(0)}f}
    \end{displaymath}
     i sigui $B_{n-1}(z)$ el producte de Blaschke de grau com a màxim $n-1$ tal que $g-B_{n-1}$ té un zero d'ordre $n-1$ en $z=0$. Llavors $zg-zB_{n-1}$ té un zero d'ordre $n$ en $0$. 
    
    Sigui 
    \begin{displaymath}
        B_n(z)= \frac{zB_{n-1}+f(0)}{1+\overline{f(0)}zB_{n-1}(z)}.
    \end{displaymath}
    Llavors $B_n(z)$ és un producte de Blaschke, $\text{grau}(B_n)=\text{grau}(zB_{n-1})\leq n$ i 
    \footnotesize
    \begin{displaymath}
        f(z)-B_n(z)=\frac{zg(z)+f(0)}{1+\overline{f(0)}zg(z)}-\frac{zB_{n-1](z)+f(0)}}{1+\overline{f(0)}zB_{n-1}(z)}=\frac{(1-\abs{f(0)}^2z(g(z)-B_{n-1}(z)))}{(1+\overline{f(0)}zg(z))(1+\overline{f(0)}zB_{n-1}(z))}.
    \end{displaymath}
    \normalsize
    Per tant $f-B_n$ té un zero d'ordre $n$ en $z=0$ i llavors \begin{displaymath}
        B_n(z)=c_0+c_1z+\dots+c_nz^n+d_{n+1}z^{n+1}.
    \end{displaymath}
\end{proof}
Per últim, un lema que ens ajudarà amb la demostració del Teorema de Pick.

\begin{lemma}
    Siguin $z_1, z_2\in\D$ diferents i siguin $ w _1, w _2\in\C$. Les següents afirmacions són equivalents:
    \begin{enumerate}[(i)]
        \item Existeix $f$ un $\D$-holomorfisme tal que $f(z_1)=w_1$, $f(z_2)=w_2$.
        \item La forma quadràtica \(\displaystyle Q_2(t_1,t_2)=\sum_{j,k=1}^2\frac{1-w_j\overline{w_k}}{1-z_j\overline{z_k}}t_j\overline{t_k}\leq0\).
        \item \(\displaystyle \abs{\frac{w_2-w_1}{1-\overline{w_1}w_2}}\leq\abs{\frac{z_2-z_1}{1-\overline{z_1}z_2}}\).
        \item \(\displaystyle \frac{(1-\abs{w_2}^2)(1-\abs{w_1}^2)}{\abs{1-\overline{w_1}w_2}^2}\geq\frac{(1-\abs{z_1}^2)(1-\abs{z_2}^2)}{\abs{1-\overline{z_1}z_2}^2}\).
    \end{enumerate}
\end{lemma}

\begin{proof} Veurem $(i)\iff(iii)\iff(iv)\iff(ii)$.

    \large
    $(i)\implies(iii)$
    \normalsize
    És el Lema de Schwarz-Pick \ref{lema:SP}.
    
    \large
    $(iii)\implies(i)$
    \normalsize
    Construïm $f=\tau_3^{-1}\circ\tau_2\circ\tau_1$ a partir de la figura (\ref{fig:lema-prev-pick}).
    \begin{displaymath}
        \tau_1=\frac{z-z_1}{1-\overline{z_1}z},\quad \tau_2=z\frac{\frac{w_2-w_1}{1-\overline{w_1}w_2}}{\frac{z_2-z_1}{1-\overline{z_1}z_2}},\quad \tau_3=\frac{w-w_1}{1-\overline{w_1}w}.
    \end{displaymath}
    $\tau_1$ i $\tau_3$ són automorfismes del disc. Com 
    \begin{displaymath}
        (iii)\iff\abs{\frac{\frac{w_2-w_1}{1-\overline{w_1}w_2}}{\frac{z_2-z_1}{1-\overline{z_1}z_2}}}\leq1,
    \end{displaymath}
    $\tau_2$ també ho és i per tant $f$ és el $\D$-holomorfisme de $(i)$.

    \begin{figure}[H]
        \centering
        \usetikzlibrary{arrows}
        \usetikzlibrary{calc}
        \begin{tikzpicture}[scale = 1.4,>=stealth']
            \draw (0,0) circle [radius=1cm];
            \draw[->] (-1.25,0) -- (1.25,0) coordinate (x axis);
            \draw[->] (0, -1.25) -- (0, 1.25) coordinate (y axis);
            \draw (1 cm,1pt) -- (1 cm,-1pt) node[anchor=north,fill=white] {1};
            \coordinate[label=right:$z_1$] (B) at (0.5, -0.3);
            \coordinate[label=right:$z_2$] (C) at (-0.4, 0.2);
            \node[right] (A) at (1.3,0) {\space};
            \node[left] (H) at (0,-1) {\space};
            \fill[black] (B) circle (1pt);
            \fill[black] (C) circle (1pt);
        
            \draw (5,0) circle [radius=1cm];
            \draw[->] (3.75,0) -- (6.25,0) coordinate (x axis);
            \draw[->] (5, -1.25) -- (5, 1.25) coordinate (y axis);
            \draw (6 cm,1pt) -- (6 cm,-1pt) node[anchor=north,fill=white] {1};
            \coordinate[label=left:{\tiny$\displaystyle\frac{z_2-z_1}{1-\overline{z_1}z_2}$}] (D) at (4.8, 0.2);
            \coordinate (E) at (5, 0);
            \node[left] (G) at (3.7,0) {\space};
            \node[right] (R) at (5,-1) {\space};
            \fill[black] (D) circle (1pt);
            \fill[black] (E) circle (1pt) node[anchor=north, fill=white] {$0$};
        
        
            \draw (0,-4) circle [radius=1cm];
            \draw[->] (-1.25,-4) -- (1.25,-4) coordinate (x axis);
            \draw[->] (0, -5.25) -- (0, -2.75) coordinate (y axis);
            \coordinate (aux1) at (1, -4);
            \draw ($(aux1) + (0, 1 pt)$) -- ($(aux1) + (0,-1pt)$) node[anchor=north,fill=white] {1};
            \coordinate[label=below:$w_1$] (I) at (-.5, -4);
            \coordinate[label=right:$w_2$] (J) at (0.3, -4.6);
            \node[right, fill=white] (K) at (1.3,-4) {\space};
            \node[left] (L) at (0,-3) {\space};
            \fill[black] (I) circle (1pt);
            \fill[black] (J) circle (1pt);
        
        
            \draw (5,-4) circle [radius=1cm];
            \draw[->] (3.75,-4) -- (6.25,-4) coordinate (x axis);
            \draw[->] (5, -5.25) -- (5, -2.75) coordinate (y axis);
            \coordinate (aux2) at (6, -4);
            \draw ($(aux2) + (0, 1 pt)$) -- ($(aux2) + (0,-1pt)$) node[anchor=north,fill=white] {1};
            \coordinate[label=left:{\tiny$\displaystyle\frac{w_2-w_1}{1-\overline{w_1}w_2} $}] (M) at (5.5, -3.3);
            \coordinate (N)  at (5, -4);
            \node[left] (P) at (3.7,-4) {\space};
            \node[right] (Q) at (5,-3) {\space};
            \fill[black] (M) circle (1pt);
            \fill[black] (N) circle (1pt) node[anchor=north, fill=white] {$0$};
        
        
            \path[->]
            (A) edge node [pos=.5, above] {$\displaystyle \tau_1=\frac{z-z_1}{1-\overline{z_1}z}$} node[pos=.5, below] {\tiny$\displaystyle\begin{aligned} z_1&\mapsto 0 \\
                z_2 &\mapsto  \frac{z_2-z_1}{1-\overline{z_1}z_2} \end{aligned}$ } (G)
            (H) edge[bend right, dashed] node [left] {$f$} (L)
            (K) edge node [pos=.5, above] {$\displaystyle \tau_3=\frac{w-w_1}{1-\overline{w_1}w}$} node[pos=.5, below] {\tiny$\displaystyle\begin{aligned} w_1&\mapsto 0 \\
                w_2 &\mapsto  \frac{w_2-w_1}{1-\overline{w_1}w_2} \end{aligned}$ } (P)
            (R) edge[bend left] node [right] {$\displaystyle \tau _{2} =z\frac{\frac{w_{2} -w_{1}}{1-\overline{w_{1}} w_{2}}}{\frac{z_{2} -z_{1}}{1-\overline{z_{1}} z_{2}}}$} node[pos=0.3, below, xshift=-1.2cm] {\tiny$\displaystyle\begin{aligned} 0&\mapsto 0 \\
                 \tau_1(z_2) &\mapsto  \frac{w_2-w_1}{1-\overline{w_1}w_2}\end{aligned}$ }(Q);
        \end{tikzpicture}

        \caption{$f=\tau_3^{-1}\circ\tau_2\circ\tau_1$}\label{fig:lema-prev-pick}
    \end{figure}

    \large
    $(iii)\iff(iv)$
    \normalsize
    És purament algebràic.
    \begin{align*}
        &\frac{\abs{w_2-w_1}}{\abs{1-\overline{w_1}w_2}}\leq\frac{\abs{z_2-z_1}}{\abs{1-z_2\overline{z_1}}} \iff 1-\abs{\frac{w_2-w_1}{1-\overline{w_1}w_2}}^2\geq1-\abs{\frac{z_2-z_1}{1-\overline{z_1}z_2}}^2\\
        &1-\abs{\frac{w_2-w_1}{1-\overline{w_1}w_2}}^2=\frac{\abs{1-\overline{w_1}w_2}^2-\abs{w_1-w_2}^2}{\abs{1-\overline{w_1}w_2}^2}=\frac{1-\abs{w_1}^2-\abs{w_2}^2+\abs{w_1w_2}^2}{\abs{1-\overline{w_1}w_2}^2}.
    \end{align*}
    D'altra banda,
    \begin{displaymath}
        \frac{(1-\abs{w_1}^2)(1-\abs{w_2}^2)}{\abs{1-\overline{w_1}w_2}^2}=\frac{1-\abs{w_1}^2-\abs{w_2}^2+\abs{w_1w_2}^2}{\abs{1-\overline{w_1}w_2}^2}.
    \end{displaymath}
    Pel mateix càlcul ho tenim per \(\displaystyle \abs{\frac{z_2-z_1}{1-\overline{z_1}z_2}}\) , per tant $(iii)\iff(iv)$.   

    \large
    $(ii)\iff(iv)$
    \normalsize
    \begin{align*}
        (ii) &= \sum_{j,k=1}^{2}\frac{1-w_j\overline{w_k}}{1-z_j\overline{z_k}}t_j\overline{t_k}\geq0\iff  
        \begin{pmatrix}
            \frac{1-\left | w_1 \right |^2}{1-\left | z_1 \right |^2}& \frac{1-w_1\overline{w_2}}{1-z_1\overline{z_2}}\\ 
            \frac{1-w_2\overline{w_1}}{1-z_2\overline{z_1}}& \frac{1-\left | w_2 \right |^2}{1-\left | z_2 \right |^2}
        \end{pmatrix}\, \text{és definida positiva.}\\
        &\iff 
        \begin{cases}
            \frac{1-\left | w_1 \right |^2}{1-\left | z_1 \right |^2}\geq0\, \text{(cert).}\\
            \text{i}\\
            \frac{1-\left | w_1 \right |^2}{1-\left | z_1 \right |^2}\frac{1-\left | w_2 \right |^2}{1-\left | z_2 \right |^2}-\abs{\frac{1-w_2\overline{w_1}}{1-z_2\overline{z_1}}}^2\geq0.
        \end{cases}
    \end{align*}
    Aquesta última desigualtat ens porta a 
    \begin{align*}
        &\frac{1-\left | w_1 \right |^2}{1-\left | z_1 \right |^2}\frac{1-\left | w_2 \right |^2}{1-\left | z_2 \right |^2}-\abs{\frac{1-w_2\overline{w_1}}{1-z_2\overline{z_1}}}^2\geq0\\
        &\iff\frac{(1-\abs{w_1}^2)(1-\abs{w_2}^2)}{(1-\abs{z_1}^2)(1-\abs{z_2}^2)}\geq\frac{\abs{1- \overline{w_1}w_2}^2}{\abs{1-\overline{z_1}z_2}^2}\\
        &\iff\frac{(1-\abs{w_1}^2)(1-\abs{w_2}^2)}{\abs{1- \overline{w_1}w_2}^2}\geq\frac{(1-\abs{z_1}^2)(1-\abs{z_2}^2)}{\abs{1-\overline{z_1}z_2}^2}=(iv).
    \end{align*}
    i amb això termina la demostració del lema.
\end{proof}

\subsection{El Problema d'interpolació}
Per fi sóm en condicions de resoldre el problema que dóna nom a aquest treball.

Siguin $\{z_1,\dots,z_n\}$ un conjunt finit de punts diferents de $\D$. Pick va determinar  $\{w_1,\dots,w_n\}$ per als quals el problema d'interpolació

\begin{equation}\label{eq:interpol}
    f(z_j)=w_j,\, \text{  } j=1,2,\dots,n
\end{equation}
%
té una solució $\D$-holomorfa $f(z)$.

\begin{theorem}[Pick]
    Existeix $f$ $\D$-holomorfisme que satisfà la interpolació \normalfont{(\ref{eq:interpol})} si i només si la forma quadràtica 
    \begin{displaymath}
        Q_n(t_1,\dots,t_n)=\sum_{j,k=1}^n=\frac{1-w_j\overline{w_k}}{1-z_j\overline{z_k}}t_j\overline{t_k}
    \end{displaymath}
    és semidefinida positiva. En aquest cas, existeix un producte de Blaschke finit de grau com a molt $n$ que resol \normalfont{(\ref{eq:interpol})}.
\end{theorem}

\begin{proof}
    Farem inducció sobre $n$. Si $n=1$, cal veure que existeix $f$ $\D$-holomorfa tal que 
    \begin{displaymath}
        f(z_1)=w_1\iff\frac{1-\abs{w_1}^2}{1-\abs{z_1}^2}\abs{t_1}^2\geq0.  
    \end{displaymath}
    
    Com $1-\abs{z_1}^2\geq0$ i $1-\abs{t_1}^2\geq0$, és equivalent dir que $\abs{w_1}\leq1$.
    
    \large
    \noindent $(\impliedby)$
    \normalsize
    Si $\abs{w_1}<1$, construïm 
    
    \begin{alignat*}{2}
    f:z &\xrightarrow{\phantom{eeeeeee}}\eqmathbox{\frac{z-z_1}{1-\overline{z_1}z}}&&\xrightarrow{\phantom{eeeeeee}}\eqmathbox{\frac{z+w_1}{1+\overline{w_1}z}}\\[-1ex]
    z_1&\xmapsto{\phantom{eeeeeee}} \eqmathbox{0} &&\xmapsto{\phantom{eeeeeee}} \eqmathbox{w_1} 
    \end{alignat*}
    %
    i $f$ és composició d'automorfismes del disc. Si $\abs{w_1}=1$, prenem $f\equiv w_1$.

    \large
    \noindent $(\implies)$
    \normalsize
    Si $\exists f$ $\D$-holomorfa tal que $f(z_1)=w_1$, llavors $\abs{w_1}\leq1$, que ja hem vist que és equivalent a $\frac{1-\abs{w_1}^2}{1-\abs{z_1}^2}\abs{t_1}^2\geq0$.

    Vist el cas $n=1$, suposem que $n > 1$. Suposem que existeix la $f$. Com és $\D$-holomorfa, llavors $\abs{w_n}\leq1$.
    Si $\abs{w_n}=1$, llavors pel principi del mòdul màxim, $f\equiv w_n$ i $w_j=w_n$ per a tot $j$.

    Suposem que $Q_n\geq0$. Si fixem $t_n=1$, $t_j=0$  $j<n$ Tenim
    \begin{displaymath}
        0\leq\sum_{j,k=1}^{n}\frac{1-w_j\overline{w_k}}{1-z_j\overline{z_k}}t_j\overline{t_k}=\frac{1-\abs{w_n}^2}{1-\abs{z_n}^2}\implies\abs{w_n}^2\leq1.
    \end{displaymath}
    
    Si $\abs{w_n}=1$, llavors fixem un $k_0$ i $t_j=0$ $\forall j\neq k_0,n$. Llavors 
    
    \noindent $Q_n(0,\dots,0,t_{k_0},0,\dots,0,t_n)$ és equivalent a $Q_2(t_{k_0},t_n)$. Pel lema anterior tenim 
    \begin{displaymath}
        \abs{\frac{w_n-w_{k_0}}{1-\overline{w_{k_0}}w_n}}\leq\abs{\frac{z_n-z_{k_0}}{1-\overline{z_{k_0}}z_n}}.
    \end{displaymath}
    Com $\frac{z-z_{k_0}}{1-\overline{z_{k_0}}z}$ és automorfisme del disc i $\abs{z_n}<1$, tenim 
    \begin{displaymath}
        \abs{\frac{z_k-z_{k_0}}{1-\overline{z_{k_0}}z_n}}<1.
    \end{displaymath}
    D'altra banda, $\frac{z-w_{k_0}}{1-\overline{w_{k_0}}z}$ és un producte de Blaschke per tant té mòdul 1 a $\partial\D$. Com $\abs{w_n}=1$, 
    \begin{displaymath}
        \abs{\frac{w_n-w_{k_0}}{1-\overline{w_{k_0}}w_n}}=1.
    \end{displaymath}
    Llavors 
    \begin{displaymath}
        1=\abs{\frac{w_n-w_{k_0}}{1-\overline{w_{k_0}}w_n}}<1.
    \end{displaymath}
    Aquest quocient no té sentit i l'única opció és $w_n=w_{k_0}$. Com aquest argument es pot aplicar a qualsevol $1\leq k_0\leq n$, tenim que $w_j=w_n$ $\forall j$. Així doncs, si $\abs{w_n}=1$, triem $f\equiv w_n$. $f$ és producte de Blaschke.
    
    Queda clar que sempre $\abs{w_j}\leq1$ i que el cas amb igualtat és trivial, per tant podem assumir que $\abs{w_n}<1$.

    Volem reduir-nos a $n-1$ per tal d'aplicar les hipòtesis d'inducció. Per a això, movem $z_n$ i $w_n$ al 0 amb automorfismes del disc (\ref{fig:pick}).

    \begin{figure}[H]
        \centering
        \begin{tikzpicture}[scale = 1.5,>=stealth']
            \draw (0,0) circle [radius=1cm];
            \draw[->] (-1.25,0) -- (1.25,0) coordinate (x axis);
            \draw[->] (0, -1.25) -- (0, 1.25) coordinate (y axis);
            \draw (1 cm,1pt) -- (1 cm,-1pt) node[anchor=north,fill=white] {1};
            \coordinate[label=right:$z_1$] (B) at (0.5, -0.3);
            \coordinate[label=right:$z_n$] (C) at (-0.4, 0.2);
            \node[right] (A) at (1.3,0) {\space};
            \node[left] (H) at (0,-1) {\space};
            \fill[black] (B) circle (1pt);
            \fill[black] (C) circle (1pt);

            \draw (5,0) circle [radius=1cm];
            \draw[->] (3.75,0) -- (6.25,0) coordinate (x axis);
            \draw[->] (5, -1.25) -- (5, 1.25) coordinate (y axis);
            \draw (6 cm,1pt) -- (6 cm,-1pt) node[anchor=north,fill=white] {1};
            \coordinate[label=left:$z'_1$] (D) at (4.8, 0.2);
            \coordinate[label=below:{$z'_n=0$}] (E) at (5, 0);
            \node[left] (G) at (3.7,0) {\space};
            \node[right] (R) at (5,-1) {\space};
            \fill[black] (D) circle (1pt);
            \fill[black] (E) circle (1pt);


            \draw (0,-3) circle [radius=1cm];
            \draw[->] (-1.25,-3) -- (1.25,-3) coordinate (x axis);
            \draw[->] (0, -4.25) -- (0, -1.75) coordinate (y axis);
            \coordinate (aux1) at (1, -3);
            \draw ($(aux1) + (0, 1 pt)$) -- ($(aux1) + (0,-1pt)$) node[anchor=north,fill=white] {1};
            \coordinate[label=below:$w_1$] (I) at (-.5, -3);
            \coordinate[label=right:$w_n$] (J) at (0.3, -3.6);
            \node[right, fill=white] (K) at (1.3,-3) {\space};
            \node[left] (L) at (0,-2) {\space};
            \fill[black] (I) circle (1pt);
            \fill[black] (J) circle (1pt);


            \draw (5,-3) circle [radius=1cm];
            \draw[->] (3.75,-3) -- (6.25,-3) coordinate (x axis);
            \draw[->] (5, -4.25) -- (5, -1.75) coordinate (y axis);
            \coordinate (aux2) at (6, -3);
            \draw ($(aux2) + (0, 1 pt)$) -- ($(aux2) + (0,-1pt)$) node[anchor=north,fill=white] {1};
            \coordinate[label=left:$w'_1$] (M) at (5.5, -2.3);
            \coordinate[label=below:{$w'_n=0$}] (N) at (5, -3);
            \node[left] (P) at (3.7,-3) {\space};
            \node[right] (Q) at (5,-2) {\space};
            \fill[black] (M) circle (1pt);
            \fill[black] (N) circle (1pt);

        
            \path[->]
            (A) edge node [pos=.5, above] {$\displaystyle z'_j=\frac{z_j-z_n}{1-\overline{z_n}z_j}$} (G)
            (H) edge[bend right] node [left] {$f$} (L)
            (K) edge node [pos=.5, above] {$\displaystyle w'_j=\frac{w_j-w_n}{1-\overline{w_n}w_j}$} (P)
            (R) edge[bend left] node [right] {$g$} (Q);
        \end{tikzpicture}
        \caption{}\label{fig:pick}
    \end{figure}
     Sigui
    \begin{displaymath}
        z_j'=\frac{z_j-z_n}{1-\overline{z_n}z_j},\; 1\leq j\leq n;\quad w_j'=\frac{w_j-w_n}{1-\overline{w_n}w_j},\; 1\leq j\leq n.
    \end{displaymath}
    Definim 
    \begin{displaymath}
        g:=\frac{f\left( \frac{z+z_n}{1-\overline{z_n}z} \right)-w_n}{1-\overline{w_n}f\left( \frac{z+z_n}{1-\overline{z_n}z} \right)}.
    \end{displaymath}
    Llavors existeix $f$ $\D$-holomorfa que resol (\ref{eq:interpol}) si i nomès si $g$ és $\D$-holomorfa i resol 
    
    \begin{equation}
        g(z_j')=w_j',\quad 1\leq j\leq n.
    \end{equation} 

    A més, $f$ és un producte de Blaschke de grau com a molt $n$ si i només si $g$ també ho és.

    D'altra banda, la forma quadràtica $Q'_n$ corresponent als punts $\{z'_1,...,z_n'\}$ i
    
    \noindent $\{w'_1,...,w'_n\}$ té una forta relació amb $Q_n$.
    Definim 
    \begin{displaymath}
        \alpha_j=\frac{(1-\abs{z_n}^2)^{1/2}}{1-\overline{z_n}z_j},\quad \beta_j=\frac{(1-\abs{w_n}^2)^{1/2}}{1-\overline{w_n}w_j}.
    \end{displaymath}
    Llavors 
    
    \begin{equation}\label{eq:zprim}
        \frac{1-z'_j\overline{z'_k}}{1-z_j\overline{z_k}}=\frac{1-\frac{z_j-z_n}{1-\overline{z_n}z_j}\overline{\left(\frac{z_k-z_n}{1-\overline{z_n}z_k}\right)}}{1-z_j\overline{z_k}}=\frac{(1-\overline{z_n}z_j)(1-z_n\overline{z_k})-(z_j-z_n)(\overline{z_k}-\overline{z_n})}{(1-z_j\overline{z_k})(1-\overline{z_n}z_j)(1-z_n\overline{z_k})}.
    \end{equation}

Veiem que \(\displaystyle \alpha_j\overline{\alpha_k}=\frac{1-\abs{z_n}^2}{(1-z_j\overline{z_n})(1-\overline{z_k}z_n)}\) és igual a (\ref{eq:zprim}):

Efectivament, 
\begin{align*}
    &\frac{(1-\overline{z_n}z_j)(1-z_n\overline{z_k})-(z_j-z_n)(\overline{z_k}-\overline{z_n})}{(1-z_j\overline{z_k})\cancel{(1-\overline{z_n}z_j)}\cancel{(1-z_n\overline{z_k})}}=\frac{1-\abs{z_n}^2}{\cancel{(1-\overline{z_n}z_j)}\cancel{(1-z_n\overline{z_k})}}\\
    \implies &\frac{1\cancel{-\overline{z_n}z_j}\cancel{-z_n\overline{z_k}}+\abs{z_n}^2z_j\overline{z_k}\cancel{+z_j\overline{z_n}}-z_j\overline{z_k}\cancel{+\overline{z_k}z_n}-\abs{z_n}^2}{(1-z_j\overline{z_k})}=1-\abs{z_n}^2\\
    \implies &\frac{1-\abs{z_n}^2+z_j\overline{z_k}\abs{z_n}^2-z_j\overline{z_k}}{(1-z_j\overline{z_k})}=\frac{(1-\abs{z_n}^2)\cancel{(1-z_j\overline{z_k})}}{\cancel{(1-z_j\overline{z_k})}}=1+\abs{z_n}^2.
\end{align*}
Pel mateix càlcul tenim \(\displaystyle \beta_j\overline{\beta_k}=\frac{1-\abs{w_n}^2}{(1-\overline{w_n}w_j)(1-\overline{w_k})}\).

Per tant, 
\begin{displaymath}
    \frac{1-w_j'\overline{w_k'}}{1-z_j'\overline{z_k'}}t_j\overline{t_k}=\frac{1-w_j\overline{w_k}}{1-z_j\overline{z_k}}\left( \frac{\beta_j}{\alpha_j} \right)t_j\overline{\left( \frac{\beta_k}{\alpha_k} \right)t_k}
\end{displaymath}
i 

\begin{equation}
    Q'_n(t_1,\dots,t_n)=Q_n\left(\frac{\beta_1}{\alpha_1}t_1,\dots,\frac{\beta_n}{\alpha_n}t_n\right).
\end{equation}

Així doncs, $Q'_n\geq0 \iff Q_n\geq0$ i hem reduït el problema al cas $z_n=w_n=0$.

Suposem, per tant, $z_n=w_n=0$. Existeix una $f$ $\D$-holomorfa tal que $f(0)=0$ i 
\begin{displaymath}
    f(z_j)=w_j,\quad 1\leq j \leq n-1
\end{displaymath}
si i només si existeix $g(z)=\frac{f(z)}{z}$ un $\D$-holomorfisme tal que 

\begin{equation}\label{eq:interpol_redux}
    g(z_j)=w_j,\quad 1\leq j\leq n-1.
\end{equation}

A més, $f$ és producte de Blaschke de grau $d$ si i només si $g$ ho és de grau $d-1$.

Per la hipòtesi d'inducció, (\ref{eq:interpol_redux}) té solució si i només si 
\begin{displaymath}
    \tilde{Q}_{n-1}(s_1,\dots,s_{n-1})=\sum_{j,k=1}^{n-1}\frac{1-\left( \frac{w_j}{z_j} \right)\overline{\left( \frac{w_k}{z_k} \right)}}{1-z_j\overline{z_k}}s_j\overline{s_k}\geq0.
\end{displaymath}
El teorema es redueix a veure que 
\begin{displaymath}
    Q_n\geq0\iff\tilde{Q}_{n-1}\geq0
\end{displaymath}
sota la condició $w_n=z_n=0$.

Com $z_n=w_n=0$, 
\begin{displaymath}
    Q_n(t_1,\dots,t_n)=\abs{t_n}^2+2\text{Re}{\sum_{j=1}^{n-1}\overline{t_j}t_n}+\sum_{j,k=1}^{n-1}\frac{1-w_j\overline{w_k}}{1-z_j\overline{z_k}}t_j\overline{t_k}.
\end{displaymath}
Si completem quadrats per a $t_n$, tenim 
\begin{displaymath}
    Q_n(t_1,\dots,t_n)=\abs{t_n+\sum_{j=1}^{n-1}t_j}^2+\sum_{j,k=1}^{n-1}\left( \frac{1-w_j\overline{w_k}}{1-z_j\overline{z_k}}-1 \right)t_j\overline{t_k}.
\end{displaymath}
Ara bé, 
\begin{displaymath}
    \frac{1-w_j\overline{w_k}}{1-z_j\overline{z_k}}-1=\frac{z_j\overline{z_k}-w_j\overline{w_k}}{1-z_j\overline{z_k}}=\frac{1-\left( \frac{w_j}{z_j} \right)\overline{\left( \frac{w_k}{z_k} \right)}}{1-z_j\overline{z_k}}z_j\overline{z_k}
\end{displaymath}
i per tant 
\begin{displaymath}
    Q_n(t_1,\dots,t_n)=\abs{\sum_{j=1}^{n}t_j}^2+\tilde{Q}_{n-1}(z_1t_1,\dots,z_{n-1}t_{n-1}).
\end{displaymath}
Llavors $\tilde{Q}_{n-1}\geq0\implies Q_n\geq0$.

\noindent Finalment, fixant \(\displaystyle t_n=-\sum_{j=1}^{n-1}t_j\) veiem que $Q_n\geq0\implies\tilde{Q}_{n-1}\geq0$.
\end{proof}
\end{document}