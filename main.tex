\documentclass[dvipsnames, svgnames, leqno, a4paper, 12pt]{article}
\usepackage{preconfig}

\title{El Problema d'interpolació de Pick--Nevanlinna}
\date{}
\author{Daniel Benages}
\begin{document}
\tikzset{font=\scriptsize}
\shorthandoff{"}
    \begin{titlepage}
      \maketitle

    \end{titlepage}

    \begin{abstract}
        L'anàlisi complexa sol destacar per l'elegància i finor dels seus teoremes i demostracions, des del Teorema de representació en sèries de potències fins al Teorema global de Cauchy. Aquest treball, però, busca resoldre un problema molt més concret. Allunyem-nos de teoremes omnipotents, de resultats que ressonen per tot el món complex. Centrem-nos en un petit disc de radi 1. Mirem dos conjunts finits de punts. Hi pot haver una funció holomorfa que els aparelli? 
        
        El camí que recorre aquest treball acaba amb el resultat que va donar Georg Pick l'any 1917. Veurem com l'anàlisi complexa, amb la seva elegància habitual, redueix el problema a l'estudi del signe d'una forma quadràtica. Això serà gràcies al Lema de Schwarz-Pick, una generalització natural del conegut Lema de Schwarz que enunciarem després de parlar d'homografies i d'automorfismes del disc. La funció que resoldrà el problema serà un producte de Blaschke finit, una classe de funcions molt útils per a la interpolació i aproximació de funcions dins $\D$.


    \end{abstract}
\newpage
\tableofcontents
\newpage
    \section{Introducció}
        L'objectiu d'aquest treball és donar un resultat sobre el següent problema d'interpolació:
        \begin{problem}\label{problema}
            Siguin $z_1,\dots,z_n$ punts diferents dins el disc unitat $\D$. Quines condicions han de complir $w_1,\dots,w_n$ per tal de que existeixi una funció holomorfa $f\colon\D\to\D$ amb
            
            \begin{equation}
                f(z_j)=w_j,\quad j=1,\dots,n
            \end{equation}
            
        \end{problem}

        La resposta ens la donarà el Teorema de Pick (1917), anomenat per Georg Pick, matemàtic austríac jueu que després de tota una vida dedicada a l'estudi de les matemàtiques va perdre la seva plaça a la Academia Txeca de les Arts i les Ciències amb l'arribada del nazisme al poder. Va morir als 82 anys al camp de concentració de Theresienstadt.


        Per poder enunciar i demostrar aquest teorema, començarem amb un breu estudi de les homografies. Aquestes ens permetran endinsar-nos en els automorfismes de \(\D\) i demostrar el famós Lema de Schwarz. Ens serà més útil una generalització feta junt amb el propi Georg Pick, el Lema de Schwarz-Pick.

        Posteriorment parlarem de l'última peça clau: els productes de Blaschke finits. Aquesta classe de funcions del disc són precisament les solucions del Problema \ref{problema}. Finalment, arribarem al teorema final:
        \begin{theorem*}[Pick]
            Existeix solució per al problema \ref{problema} 
            si i només si la forma quadràtica \begin{displaymath}
                Q_n(t_1,\dots,t_n)=\sum_{j,k=1}^n=\frac{1-w_j\overline{w_k}}{1-z_j\overline{z_k}}t_j\overline{t_k}
            \end{displaymath}
            és semidefinida positiva. En aquest cas, existeix un producte de Blaschke finit de grau com a molt $n$ que resol el problema \ref{problema}.
        \end{theorem*}

        Aquesta tasca, però, és inabordable sense fer algunes concessions. La més important és que donarem per fetes la majoria de nocions que s'obtindrien en un curs elemental d'anàlisi complexa. Si més no, fins al teorema del mòdul màxim (tot i que l'enunciarem per refrescar la memòria).



    \section{Homografies}
    Aquesta secció del treball està basada en les primeres seccions del capítol 1 del llibre \cite{ford_1972}, amb alguns resultats del capítol 14 de \cite{rudin_1974}.
        Sigui  \(\Omega\) un domini de \(\mathbb{C}\) i \(f: \Omega\to\Omega\) una funció holomorfa, diem que \(f\) és \textbf{conforme} si \(f'(z) \neq 0\,\  \forall z\in\Omega\).
        
        Diem \textbf{esfera de Riemann} a la compactificació del pla complex per un punt. La manera usual de pensar en aquest espai topològic és considerar una esfera on el pol nord és \(\infty\). La denotem per \(S^2\).

        \begin{figure}[H]
            \centering

            \begin{tikzpicture}
                \coordinate (aux) at (0, -2);
                \coordinate (aux2) at (0, 2);
                \shade[ball color = gray!60, opacity = 0.6] (0,0) circle (2cm);
                \draw (0,0) circle (2cm);
                \draw (-2,0) arc (180:360:2 and 0.6) node[pos=.5] (A) {};
                \draw ($(A) + (0, 1pt)$) -- ($(A) + (0,-1pt)$) node [pos=.5, below]{$-i$};
                \draw[dashed] (2,0) arc (0:180:2 and 0.6) node[pos=.5] (B) {};
                \draw ($(B) + (0, 1.1pt)$) -- ($(B) + (0,-1.1pt)$) node [pos=.5, above]{$i$};
                \draw[cm={cos(90) ,-sin(90) ,sin(90) ,cos(90) ,(0 cm,0 cm)}] (2,0) arc (0:180:2 and 0.6);
                \draw[dashed,cm={cos(90) ,-sin(90) ,sin(90) ,cos(90) ,(0 cm,0 cm)} ] (-2,0) arc (180:360:2 and 0.6);
                \fill[fill=black] (0,0) circle (1pt);
                \draw (2 cm,1pt) -- (2 cm,-1pt) node[anchor=west] {1};
                \draw (-2 ,1pt) -- (-2,-1pt) node[anchor=east] {-1};
                \draw ($(aux) + (0,1pt)$)-- ($(aux) + (0,-1pt)$) node[anchor=north] {0};
                \draw ($(aux2) + (0,1pt)$)-- ($(aux2) + (0,-1pt)$) node[anchor=south] {$\infty$};
              \end{tikzpicture}

            \caption{L'esfera de Riemann.}
        \end{figure}
        No ens cal preocupar-nos per les propietats topològiques de \(S^2\). Per al que a nosaltres ens ocupa, l'esfera de Riemann es comporta com \(\mathbb{C}\), però ens permet tractar \(\infty\) estalviant-nos límits. Entendrem que \(1/0 = \infty\) i que \(1/\infty = 0\).

        De fet, aquest és el motiu principal per introduir el concepte de \(S^2\), ja que serveix de "cèrcol" per no deixar escapar els punts que habitualment hauríem de considerar pols de funcions altrament holomorfes.

        Siguin \(a,b,c,d\in \C\). Una \textbf{homografia} és una aplicació de la forma
        \begin{displaymath}
            T(z)=\frac{az+b}{cz+d},\quad ad-bc\neq0.
        \end{displaymath}
        Observem que $T$ és holomorfa a $\C\mysetminus\{-d/c\}$ i $ad-bc\neq0$ ens dona conformitat.
        
        \begin{proposition}
            Les homografies són invertibles i la seva inversa és una homografia.
        \end{proposition}
        
        \begin{proof}
            Si \begin{math}
                 w  = \frac{az+b}{cz+d} 
            \end{math}, podem aïllar $z$ i obtenim \begin{math}
                z = \frac{-d  w +b}{c w -a}
            \end{math}.
        \end{proof}

        \begin{corollary}
            Les homografies són bijeccions meromorfes de $\C$.
        \end{corollary}

        \begin{lemma}
            Si una funció $f$ té un pol d'ordre igual o superior a 2, $f$ no pot ser injectiva.  
        \end{lemma}

        \begin{proof}
            Considerem la funció $1/f$. Com $f$ és injectiva si i només si $1/f$ ho és. És suficient demostrar la no injectivitat de $1/f$.
            
            Sigui $a$ un pol d'ordre $m>1$ de $f$. Llavors $a$ és un zero d'ordre $m$ de $1/f$, per tant 
            \begin{displaymath}
                \frac{1}{f}=(z-a)^mh(z),\quad  h(a)\neq0
            \end{displaymath} 
            per a una certa funció holomorfa $h$ en un entorn del punt $a$.

            Considerem $ g(z)=(z-a)h(z)^{1/m}$. Aquesta funció és holomorfa i ben definida a un entorn del punt $a$. La seva derivada és 
            \begin{displaymath}
                g'(z)=\frac{1}{m}(z-a)h(z)^{\frac{1}{m}-1}h'(z)+h(z)^{\frac{1}{m}}
            \end{displaymath} 
            per tant $g'(a)\neq0$.
            Tenim, doncs, un entorn de $a$ on $g$ és una funció holomorfa i invertible que envia $a$ al zero. Amb el canvi de coordenades $ w  = g(z)$, veiem que localment 
            \begin{displaymath}
                 w ^m = \frac{1}{f(g^{-1}(z))}
            \end{displaymath} 
            per tant en un entorn de $a$ la funció $1/f$ es comporta com la funció $z\to z^m$, que per $m>1$ no és injectiva.
        \end{proof}

        \begin{theorem}
            Sigui una $f\colon S^2\to S^2$ una bijecció conforme a tot $S^2$ tret, potser, d'un nombre finit de punts. Llavors $f$ és una homografia.
        \end{theorem}

        \begin{proof}
            Sigui $f$ aquesta bijecció i $q_1,\dots,q_n$ els punts exclòsos. Com $f$ és conforme, $f$ és holomorfa tret de en $q_i$. Aquests punts només poden ser pols, singularitats essencials o evitables. 
            
            La possibilitat de que siguin essencials queda descartada, ja que pel Gran Teorema de Picard, en tot entorn d'una discontinuïtat d'aquest tipus la funció $f$ pren com a valors tots els punts de $\C$ un nombre infinit de vegades, per tant $f$ no podria ser bijectiva. Com les discontinuïtats de $f$ són o evitables o pols, podem assegurar que $f$ es una funció racional. 
            
            A més, per la injectivitat a $S^2$, només un dels punts pot ser un pol, el qual pel lema anterior seria d'ordre 1. Si aquest es troba en un punt finit $q_k$, \begin{displaymath}
                f(z)=\frac{A_1}{z-q_k}+A_0=\frac{A_0z+A_1-A_0q_k}{z-q_k}\,\text{, } A_1\neq0.
            \end{displaymath}
            Si $q_k=\infty$, 
            \begin{displaymath}
                f(z)=A_1z+A_0\, \text{, } A_1\neq0.
            \end{displaymath}
            Sigui com sigui, la funció és una homografia.
        \end{proof}

        \begin{proposition}\label{prop:comp_homo}
            La composició finita d'homografies és una sola homografia.
        \end{proposition}

        \begin{proof}
            És suficient demostrar-ho per la composició de dues homografies. Siguin 
            \begin{displaymath}
                T(z)=\frac{az+b}{cz+d},
            \end{displaymath}
            \begin{displaymath}
                S(z)=\frac{a'z+b'}{c'z+d'}.
            \end{displaymath}
            Llavors
            \begin{displaymath}
                (T\circ S)(z)=\frac{a\frac{a'z+b'}{c'z+d'}+b}{c\frac{a'z+b'}{c'z+d'}+d}=\frac{aa'z+ab'+bc'z+bd'}{ca'z+cb'+dc'z+dd'}=\frac{\left( aa'+bc' \right)z+(ab'+bd')}{\left( ca'+dc' \right)z+(cb'+dd')},
            \end{displaymath}
            \normalsize
            que clarament és una homografia.
        \end{proof}

        Degut a això, podem descompondre totes les homografies en combinacions de tres classes fonamentals:
        
        \begin{theorem}\label{th:deschomo}
            Tota homografia és composició finita de translacions, rotacions, homotècies i inversions.
        \end{theorem}
        
        \begin{proof}
            Totes aquestes transformacions són clarament homografies. Veiem que podem crear una cadena de composicions que ens porti a qualsevol homografia:
            
            Si $c=0$, simplement $z\to az\to az+b\to \frac{az+b}{d}$. Si $c\neq0$, llavors 
            \begin{align*}
                &z\to cz\to cz+d\to \frac{1}{cz+d}\to \frac{\frac{bc-ad}{c}}{cz+d}\\
                &\to \frac{\frac{bc-ad}{c}}{cz+d}+\frac{a}{c}=\frac{bc-ad}{c(cz+d)}+\frac{a}{c}=\frac{bc-ad+azc+ad}{c(cz+d)}=\frac{az+b}{cz+d}.
            \end{align*}
            En qualsevol dels casos, arribem a l'homografia.
        \end{proof}

        \noindent Definim la \textbf{raó doble} entre quatre nombres complexos $z_1,z_2,z_3,z_4$ per 
        \begin{displaymath}
                (z_1,z_2,z_3,z_4) = \frac{(z_1-z_3)(z_2-z_4)}{(z_2-z_3)(z_1-z_4)}.
        \end{displaymath}
            %
        Ens serà útil considerar que algun d'aquests punts sigui $\infty$. En aquest cas, ometrem els termes que l'incloguin. Per exemple, si $z_1 = \infty$, llavors 
        \begin{displaymath}
                (\infty, z_2,z_3,z_4) = \frac{z_2-z_4}{z_1-z_3}.
        \end{displaymath}

        \noindent Donats $z_1,z_2,z_3\in\mathbb{C}$, la funció donada per la raó doble 
        
        \begin{equation}\label{eq:rao_doble}
            (z,z_1,z_2,z_3) = \frac{(z-z_2)(z_1-z_3)}{(z-z_3)(z_1-z_2)}=\frac{(z_1-z_3)z-z_2(z_1-z_3)}{(z_1-z_2)z-z_3(z_1-z_2)}
        \end{equation}
        %
        és una homografia que envia $z_1\to1$, $z_2\to0$ i $z_3\to\infty$.

        A més, si algun $z_i$ és $\infty$, utilitzant la raó doble que pertoca es segueix complint aquesta afirmació.

    \begin{proposition}
        Les homografies conserven la raó doble.
    \end{proposition}

    \begin{proof}
        % Siguin $z_1,z_2,z_3,z_4\in \C$ (finits); $z'_1,z'_2,z'_3,z'_4$ la seva imatge per una homografia que no envii cap a $\infty$. Tenim 
        % \footnotesize
        % \begin{align*}
        %     z'_1-z'_3 &=\frac{az_1+b}{cz_1+d}-\frac{az_3+b}{cz_3+d}\\
        %     &= \frac{acz_1z_3+adz_1+bcz_3+bd-acz_1z_3-adz_3-bcz_1-bd}{(cz_1+d)(cz_3+d)}
        %     = \frac{(ad-bc)(z_1-z_3)}{(cz_1+d)(cz_3+d)}.
        % \end{align*}
        % \normalsize
        % Ídem per a $z'_1-z'_4, z'_2-z'_4, z'_2-z'_3$, per tant un cop simplifiquem tenim 
        % \begin{displaymath}
        %     \frac{(z'_1-z'_3)(z'_2-z'_4)}{(z'_1-z'_4)(z'_2-z'_3)}=\frac{\frac{(ad-bc)^2(z_1-z_3)(z_2-z_4)}{(cz_1+d)(cz_2+d)(cz_3+d)(cz_4+d)}}{\frac{(ad-bc)^2(z_1-z_4)(z_2-z_3))}{(cz_1+d)(cz_2+d)(cz_3+d)(cz_4+d)}} = \frac{(z_1-z_3)(z_3-z_4)}{(z_1-z_4)(z_2-z_3)}.
        % \end{displaymath}
        % Per últim, suposem un dels punts és $\infty$. Per exemple, sigui $z'_1=\infty$, tenim que $d=-cz_1$. La raó doble de les imatges és 
        % \begin{align*}
        %     (\infty,z'_2,z'_3,z'_4) &=\frac{(z'_2-z'_4)}{(z'_2-z'_3)}=\frac{(z_2-z_4)(cz_3+d)}{(z_2-z_3)(cz_4+d)}=\frac{(z_2-z_4)(cz_3-cz_1)}{(z_2-z_3)(cz_4-cz_1)}\\
        %     &=\frac{(z_1-z_3)(z_2-z_4)}{(z_1-z_4)(z_2-z_3)},
        % \end{align*}
        % per tant es conserva.
        Pel Teorema \ref{th:deschomo}, és suficient veure que per a tota inversió es conserva la raó doble. És a dir, que 
        \begin{displaymath}
            \left( \frac{1}{z_1}, \frac{1}{z_2}, \frac{1}{z_3}, \frac{1}{z_4} \right)=\left( z_1, z_2, z_3, z_4 \right),\quad \text{per a tot }\, z_1,z_2,z_3,z_4\in\C.
        \end{displaymath}
        Això es comprova fent un càlcul senzill.
    \end{proof}

    \begin{theorem}\label{th:homounic}
        L'única homografia que deixa fixos més de dos punts és la identitat.
    \end{theorem}

    \begin{proof}
        Donada l'homografia $\frac{az+b}{cz+d}$, els seus punts fixos compleixen 
        \begin{displaymath}
            \frac{az+b}{cz+d}=z\implies cz^2+(d-a)z+b=0.
        \end{displaymath}
        Si hi ha més de dos punts fixos, deduïm que $c=b=0$, $d=a$ i, per tant, l'aplicació és la identitat.
    \end{proof}

    Això ens porta al teorema de caracterització de les homografies.

    \begin{theorem}
        Tota homografia està definida per les imatges de tres punts diferents.
        Dit d'una altra manera, donats $z_1,z_2,z_3,z'_1,z'_2,z'_3\in\C$, existeix una única homografia $T$ tal que $T(z_1)=z'_1$, $T(z_2)=z'_2$, $T(z_3)=z'_3$.
    \end{theorem}

    \begin{proof}
        Demostrem primer la unicitat. Siguin $S$, $T$ dues homografies tals que $S(z_i)=T(z_i)=z'_i$. Llavors, per Proposició \ref{prop:comp_homo}, $S^{-1}T$ és una homografia que compleix 
        \begin{displaymath}
            S^{-1}T(z_i)=S^{-1}(z'_i)=z_i\, i=1,\dots3.
        \end{displaymath}
        Com té tres punts fixos, pel Teorema \ref{th:homounic}  $S^{-1}T=id$. Per tant, $T=S$.

        Per veure l'existència, recordem que per (\ref{eq:rao_doble}), la rao doble $(z,z_1,z_2,z_3)$ ens envia $z_1\to1,\, z_2\to0,\, z_3\to\infty$. 

        Siguin $S(z)=(z,z'_1,z'_2,z'_3)$ i $T(z)=(z,z_1,z_2,z_3)$
        Llavors, la composició 
        \begin{displaymath}
            S^{-1}\circ T
        \end{displaymath}
        és la homografia que busquem.
    \end{proof}
    
    Per últim, estudiem les rectes i circumferències a $\C$. Donates $p,q\in\C$, la mediatriu del segment que va de $p$ a $q$ és 
    \begin{displaymath}
        \abs{z-p}=\abs{z-q}.
    \end{displaymath}
    Elevant al quadrat i reordenant termes,  obtenim 
    \begin{displaymath}
        (\bar{p}-\bar{q})z+\bar{z}(p-q)+\abs{p}^2+\abs{q}^2=0
    \end{displaymath}
    Diem $\beta=\bar{p}-\bar{q}$, $\gamma=\abs{p}^2-\abs{q}^2$, les rectes del pla són de la forma
    \begin{equation}
        \beta z+\bar{\beta}\bar{z}+\gamma=0.
    \end{equation}

    Veiem que les circumferències són de la forma 
    \begin{displaymath}
        \abs{z-p}=\rho\abs{z-q},
    \end{displaymath}
    per $\rho\neq1$. Sigui $z=w+q$, $c=p-q$. Llavors 
    \begin{displaymath}
        \abs{w-c}=\rho\abs{w}.
    \end{displaymath}
    Com abans, quadrats i reordenant
    \begin{equation}\label{eq:circ}
        (1-\rho^2)\abs{w}^2-w\bar{c}-\bar{w}c+\abs{c}^2=0.
    \end{equation}
    Completem quadrats
    \begin{align*}
        0 &=(1-\rho^2)\abs{w}^2-w\bar{c}-\bar{w}c+\abs{c}^2\\
        &=\abs{w}^2-\frac{w\bar{c}-\bar{w}c}{(1-\rho^2)}+\frac{\abs{c}^2(1-\rho^2)}{(1-\rho^2)^2}
    \end{align*}
    i per tant
    \begin{align*}
        &\abs{w}^2-\frac{w\bar{c}-\bar{w}c}{(1-\rho^2)}+\frac{\abs{c}^2}{(1-\rho^2)^2}=\frac{\abs{c}^2\rho^2}{(1-\rho^2)^2}\\
        &\implies \abs{w-\frac{c}{1-\rho^2}}=\abs{c}\frac{\rho}{1-\rho^2},
    \end{align*}
    que és una circumferència. Siguin $\alpha=1-\rho^2$, $\beta=-c$, $\gamma=\abs{c}^2$, a (\ref{eq:circ}) tenim
    \begin{equation}\label{eq:recticirc}
        \alpha z\bar{z}+\beta z+ \bar{\beta}\bar{z}+\gamma=0,\quad \alpha,\gamma\in\mathbb{R}, \beta\in\C.
    \end{equation}
    Com el radi ha de ser positiu, és necessari que $\abs{\beta}^2>\alpha\gamma$. 
    
    Tenim doncs que si $\alpha=0$, (\ref{eq:recticirc}) ens l'equació d'una recta. Si $\alpha\neq0$, (\ref{eq:recticirc}) és l'equació d'una circumferència.

    \begin{theorem}
        Les homografies envien rectes i circumferències a rectes i circumferències.
    \end{theorem}
    \begin{proof}
        És suficient veure-ho per translacions, rotacions, homotècies i inversions. Les tres primeres és evident que envien rectes a rectes i circumferències a circumferències.  Les inversions no, però sigui $w=\frac{1}{z}$, podem veure que fer la inversió a (\ref{eq:recticirc}) ens dona 
        \begin{equation*}
            \alpha w\bar{w}+\beta w+ \bar{\beta}\bar{w}+\gamma=0,
        \end{equation*}
        que és una equació del mateix tipus.
    \end{proof}
\section[Automorfismes al Disc Unitat]{Automorfismes a $\D$}
    Aquesta secció del treball conté resultats de la secció 12 del primer capítol del llibre \cite{ford_1972}, del capítol 12 de \cite{rudin_1974} i del primer capítol de \cite{dineen_2016}, amb les demostracions adaptades a $\D$.
    Com ja hem comentat prèviament, centrarem la nostra atenció exclusivament a $\D$. Per tant, els teoremes i proposicions que enunciarem a partir d'ara seran sempre en aquesta regió. Recordem que definim com Disc Unitat
    
    \begin{equation*}
            \D:=\{\, z\in\C\colon |z|<1\, \}.
    \end{equation*}

    \noindent Observem que $\D$ és un domini acotat de $\C$.

    Enunciem ara el Teorema del Mòdul Màxim que, tot i que no el demostrarem, ens serà indispensable per als propers resultats. 
    
    \begin{theorem}[Teorema del Mòdul Màxim]\label{th:TMM}
        Sigui $K$ la clausura d'una regió acotada $\Omega$. Si $f$ és continua en $K$ i holomorfa en $\Omega$, llavors \begin{equation}
            \abs{f(z)}\leq\sup\{\abs{f(w)}\colon w\in\partial\Omega\},\;\text{per a tot } z\in\Omega.
        \end{equation}
        Si es dona la igualtat per a algun $z\in\Omega$, llavors $f$ és constant.
    \end{theorem}
    
    Anomenarem $\mathbbb{D}$\textbf{-holomorfisme} o funció $\mathbbb{D}$\textbf{-holomorfa} a totes aquelles funcions  holomorfes que envien $\D$ en si mateix.
    
    Enunciem ara el Teorema de caracterització dels automorfismes holomorfs de $\D$:
    
    \begin{theorem}
        Una funció $T$ $\D$-holomorfa és bijectiva si i només si
        
        \begin{equation}
            T(z)=\lambda\frac{a-z}{1-\overline{a}z},\, \text{amb }a\in\D\; \text{ i }\; \abs{\lambda}=1.
        \end{equation}

    \end{theorem}

    Dit d'una altra manera, els automorfismes del disc són una classe molt concreta d'homografies.

    \begin{proof}
        Pel que ja hem vist, sabem que $T(z)=\lambda\frac{a-z}{1-\overline{a}z}$ és automorfisme holomorf de $\C$. Només cal veure que és bijectiu a $\D$. 

        Sigui $z\in\D$. Tenim
        \begin{align*}
            \frac{\abs{a-z}}{\abs{1-\overline{a}z}}<1 &\iff \abs{a-z}^2<\abs{1-\overline{a}z}^2\\
            &\iff\abs{a}^2+\abs{z}^2-2\Re{a\overline{z}}<1+\abs{\overline{a}z}^2-2\Re{\overline{\overline{a}z}}\iff\\
            &\iff\left( \abs{z}^2-1 \right)\left( 1-\abs{a}^2 \right)<0\\
            &\iff\abs{a}<1.
        \end{align*}
        Per tant $\D\xrightarrow{T} \D$ és automorfisme holomorf.

        Vegem ara la implicació contraria. Sigui $T$ un $\D$-holomorfisme bijectiu. 
        Suposem primer que $T(0)=0$. Com $T$ és holomorfa, és una serie de potencies i si deixa el 0 fix, $\frac{T(z)}{z}$ és holomorfa en $\D$. Considerem $\abs{\frac{T(z)}{z}}$ en un disc $U$ de radi $r<1$ centrat a l'origen. Pel Teorema del Mòdul Màxim \ref{th:TMM}, $\abs{T(z)/z}$ és màxim en $\abs{z}=r$. Llavors en $U$ tenim 
        \begin{displaymath}
            \abs{\frac{T(z)}{z}}\leq\frac{1}{r}.
        \end{displaymath}
        Fent tendir $r\to1$, deduïm que $\abs{T(z)}\leq\abs{z}$.
        Sigui $S$ la inversa de $T$. Pel mateix raonament tenim $\abs{\frac{S(z)}{z}}\leq1$ per a tot $z\in\D$. Si notem $z'=T(z)$, llavors
        \begin{displaymath}
            1\geq\abs{\frac{S(z')}{z'}}=\abs{\frac{z}{T(z)}}\implies \abs{\frac{T(z)}{z}}=1,\; \text{per a tot } z\in\D.
        \end{displaymath}
        i com el valor absolut és constant, també ho és la funció (pel fet de ser holomorfa). Llavors 
        
        \begin{equation}
            \frac{T(z)}{z}=e^{i\alpha}\implies T(z)=e^{i\alpha}z
        \end{equation}
        %
        per una $\alpha\in\mathbb{R}$ constant. Per tant, un automorfisme bijectiu de $\D$ que fixa l'origen és una rotació. 

        Eliminem la restricció de que l'origen sigui fix. Sigui $T(z)=a\neq0$. Considerem l'homografia 
        \begin{displaymath}
            R=\lambda\frac{a-z}{1-\overline{a}z}
        \end{displaymath}
        amb $\lambda=e^{i\beta}$, $\beta\in\mathbb{R}$. La composició $R^{-1}T$ és un automorfisme holomorf de $\D$ que deixa fix l'origen. Per tant és una rotació, diguem-li $e^{i\alpha}$. 

        Llavors 
        \begin{displaymath}
            R^{-1}T=e^{i\alpha}\implies T=Re^{i\alpha}\implies T(z)=e^{i(\alpha+\beta)}\frac{a-z}{1-\overline{a}z}=\lambda'\frac{a-z}{1-\overline{a}z}.
        \end{displaymath}
    \end{proof}
    
    A partir d'ara, denotarem 
    
    \begin{equation}
        \varphi_a(z):=\frac{a-z}{1-\overline{a}z}.
    \end{equation}

    Vegem ara un resultat clàssic de l'anàlisi complexa.

    \begin{theorem}[Lema de Schwarz]\label{th:sch}
        Sigui $f:\D\to\D$ una funció holomorfa amb $f(0)=0$.

        Aleshores
        \begin{enumerate}[(i)]
            \item $\abs{f(z)}\leq\abs{z}$ per a tot $z\in\D$.
            \item $\abs{f'(0)}\leq 1$.
        \end{enumerate}
        A més, si hi hagués igualtat en algun dels dos casos, llavors $f(z)=e^{i\alpha}z$ per a algun $\alpha\in\mathbb{R}$ i per a tot $z\in\D$.
    \end{theorem} 

    \begin{proof}
        Demostrarem primer \textit{(i)}.

        Com $f(0)=0$ i és holomorfa, tenim que $h(z)=f(z)/z$ és holomorfa en tot $\D$. Pel Teorema del mòdul màxim \ref{th:TMM}, 
        
        \begin{equation}
            \sup_{\abs{z}\leq r}\abs{h(z)}=\sup_{\abs{z}= r}\abs{h(z)}=\frac{1}{r}\sup_{\abs{z}\leq r}\abs{f(z)}
        \end{equation}
        %
        per a $0<r<1$. Com $\abs{f(z)}\leq 1$ per a tot $z\in\D$, fem tendir $r$ a 1 i tenim 
        
        \begin{equation}
            \sup_{z\in\D}\abs{h(z)}\leq 1\implies \abs{f(z)}\leq \abs{z}.
        \end{equation}
        
        Veiem també que si $\abs{f(z_0)}=\abs{z_0}$ per a algun $z_0\in\D\setminus\{0\}$, llavors $\abs{h(z_0)}=1$ i, pel teorema del mòdul màxim, $h$ és constant de mòdul 1. Així, existeix $\alpha\in\mathbb{R}$ tal que 
        \begin{displaymath}
            f(z)=zh(z)=ze^{i\alpha}, \; \forall z\in\D.
        \end{displaymath}
        
        Vegem ara \textit{(ii)}.

        Com $f(z)/z=h(z)$ i $f(0)=0$, llavors 
        
        \begin{equation}
            \abs{f'(0)}=\lim_{z\to0}\frac{\abs{f(z)}}{\abs{z}}=\lim_{z\to0}\abs{h(z)}=\abs{h(0)}\leq1.
        \end{equation} 
    
        A més, si $\abs{f'(0)}=\abs{h(0)}=1$, pel teorema del mòdul màxim un altre cop tenim $h=e^{i\alpha}$ per a un cert $\alpha\in\mathbb{R}$ i per tant $f(z)=e^{i\alpha}z$, per a tot $z\in\D$.
    \end{proof}

    No tan famosa és una generalització d'aquest lema, on es relaxen les hipòtesis i no es suposa que $f(0)=0$. Aquest és el lema de Schwarz-Pick:
    
    \begin{theorem}[Lema de Schwarz-Pick]\label{lema:SP}
        Sigui $f\colon\D\to\D$ una funció holomorfa. Llavors
        \begin{enumerate}[(i)]
            \item \(\displaystyle \abs{\frac{f(z)-f( w )}{1-f(z)\overline{f( w )}}}\leq\abs{\frac{z- w }{1-z\overline{ w }}},\text{ per a tot }z, w \in\D\).
            \item \(\displaystyle \frac{\abs{f'(z)}}{1-\abs{f(z)}^2}\leq\frac{1}{1-\abs{z}^2}\text{ per a tot }z\in\D\).
        \end{enumerate}
        La igualtat en tots dos casos es dona si $f$ és automorfisme de $\D$. Si la igualtat de \textit{(i)} es compleix per a uns $z, w$ amb $z\neq w $, o si es dona en \textit{(ii)} per a algun $z$, llavors $f$ és automorfisme de $\D$.
    \end{theorem} 

    \begin{proof}
        Sigui $g = \varphi_{f( w )}\circ f\circ \varphi_{- w }$. En particular tenim $g(0)=\varphi_{f( w )}\left( f( w ) \right)=0$. Pel lema de Schwarz \ref{th:sch}, $\abs{g(\xi)}\leq\abs{\xi}$ per a tot $\xi\in\D$ i deduïm que 
        \begin{displaymath}
            \abs{\varphi_{f( w )}(f(z))}\leq\abs{\varphi_ w (z)}.
        \end{displaymath}
        Substituint cada terme per la seva definició 
        
        \begin{equation}
            \abs{\frac{f(z)-f( w )}{1-\overline{f( w )}f(z)}}\leq\abs{\frac{z- w }{1-\overline{ w }z}}.
        \end{equation}
        
        Si tenim igualtat per uns $z\neq w $, llavors $\abs{g(\varphi_ w (z))}=\abs{\varphi_ w (z)}$ i pel lema de Schwarz, $g(z)=ze^{i\alpha}$.Per tant, $g$ és un automorfisme de $\D$ i deduïm que  $f=\varphi_{-f( w )}\circ g\circ \varphi_ w $ també ho és.

        Veiem ara \textit{(ii)}.

        \begin{displaymath}
            \frac{\abs{f'(z)}}{1-\abs{f(z)}^2}=\lim_{ w \to z}\left( \abs{\frac{f(z)-f( w )}{z- w }}\frac{1}{\abs{1-\overline{f( w )}f(z)}} \right)\leq\lim_{ w \to z}\abs{\frac{1}{1-\overline{ w }z}}=\frac{1}{1-\abs{z}^2}.
        \end{displaymath}

        Si tenim igualtat a \textit{(ii)}, considerem $z= w $ i com \(\displaystyle \varphi_{a}'(z)=\frac{1-\abs{a}^2}{(1-\overline{a}z)^2}\), tenim 
        \begin{align*}
            \abs{g'(0)} &= \abs{\varphi_{f( w )}'(f\circ\varphi_{- w }(0))}\abs{f'\left( \varphi_{- w }(0) \right)}\abs{\varphi_{- w }'(0)}\\
            &=\abs{\varphi_{f( w )}'\left( f( w ) \right)}\abs{f'( w )}\left( 1-\abs{ w }^2 \right)\\
            &=\frac{\abs{f'( w )}}{1-\abs{f( w )}^2}\left( 1-\abs{ w }^2 \right)=1
        \end{align*}
        i, pel Lema de Schwarz, $g$ és automorfisme i $f=\varphi_{-f( w )}\circ g\circ \varphi_ w $ també. 

        Finalment, si $f$ és automorfisme de $\D$, apliquem \textit{(i)} a $f$ i $f^{-1}$ i obtenim 
        
        \begin{equation}
            \abs{\frac{z- w }{1-z\overline{ w }}}=\abs{\frac{f^{-1}(f(z))-f^{-1}(f( w ))}{1-f^{-1}(f(z))\overline{f^{-1}(f( w ))}}}\leq\abs{\frac{f(z)-f( w )}{1-f(z)\overline{f( w )}}}\leq\abs{\frac{z- w }{1-z\overline{ w }}}
        \end{equation} 
        %
        per tant hi ha igualtat.
        De la mateixa manera, per a \textit{(ii)} tenim 
        
        \begin{equation}
            \frac{1}{1-\abs{z}^2}=\frac{\abs{(f^{-1}\circ f)'(z)}}{1-\abs{(f^{-1}\circ f)(z)}^2}\leq\frac{f'(z)}{1-\abs{f(z)}^2}\leq\frac{1}{1-\abs{z}^2}
        \end{equation} 
        
        aconseguint novament la igualtat.
    \end{proof}



\section{El Teorema de Pick}
Aquesta part del treball està basada en les seccions 1 i 2 del llibre \cite{garnett_2007}.
\subsection{Productes de Blaschke finits}

L'última peça clau per al problema d'interpolació són els productes de Blaschke. No ens caldrà anar més enllà del cas finit.

Un \textbf{producte de Blaschke finit} és una funció de la forma 
\begin{displaymath}
    B(z)=e^{i\alpha}\prod_{j=1}^{n}\frac{z-z_j}{1-\overline{z_j}z},\;\ \abs{z_j}<1.
\end{displaymath}
Ens seran útils les següents propietats:

\begin{proposition}
    Si $B$ és un producte de Blaschke, és compleix:
    \begin{enumerate}[(i)]
        \item $B$ és continua en $\partial\D$.
        \item $\abs{B}=1$ en $\partial\D$.
        \item $B$ té un nombre finit de zeros en $\D$.
    \end{enumerate}
    Aquestes propietats, junt amb on són aquestes zeros, determinen $B$ tret d'una constant de mòdul 1.
\end{proposition}

\begin{proof}
   Les propietats $(i)$, $(ii)$, $(iii)$ són evidents. Veurem la part final.

    Sigui una funció $f$ que compleix les tres i $B$ un producte de Blaschke amb els mateixos zeros. Pel principi del mòdul màxim, $\abs{f/B}\leq1$ i $\abs{B/f}\leq1$ en $\D$. Per tant $f/B$ és constant.
\end{proof}

Els productes de Blaschke, a més de ser útils en la interpolació (com veurem pròximament),  juguen un paper important en l'aproximació de funcions del disc. 

\begin{theorem}[Carathéodory]
    Sigui $f$ un $\D$-holomorfisme. Existeix $\{B_k\}$ una successió de productes de Blaschke que convergeixen puntualment a $f$.
\end{theorem}

\begin{proof}
    Com $f$ és holomorfa, és una serie de potencies. Siguin $\{c_k\}$ els seus coeficients. Demostrarem el teorema per inducció, trobant un producte de Blaschke amb com a molt $n$ zeros tal que els $n$ primers coeficients coincideixen amb els de $f$, i.e., 
    \begin{displaymath}
        B_n=c_0+c_1z+\dots+c_{n-1}z^{n-1}+d_nz^n\dots
    \end{displaymath}
    Com $\abs{c_0}\leq1$, podem fixar $B_0=\frac{z+c_0}{1+\overline{c_0}z}$. Si $\abs{c_0}=1$, llavors $B_0=c_0$ és un producte de Blaschke de grau 0.

    Suposem que per a qualsevol $g$ $\D$-holomorfisme tenim construït el seu $B_{n-1}(z)$. Sigui 
    \begin{displaymath}
        g=\frac{1}{z}\frac{f-f(0)}{1-\overline{f(0)}f}
    \end{displaymath}
     i sigui $B_{n-1}(z)$ el producte de Blaschke de grau com a màxim $n-1$ tal que $g-B_{n-1}$ té un zero d'ordre $n-1$ en $z=0$. Llavors $zg-zB_{n-1}$ té un zero d'ordre $n$ en $0$. 
    
    Sigui 
    \begin{displaymath}
        B_n(z)= \frac{zB_{n-1}(z)+f(0)}{1+\overline{f(0)}zB_{n-1}(z)}.
    \end{displaymath}
    Llavors $B_n(z)$ és un producte de Blaschke, $\text{grau}(B_n)=\text{grau}(zB_{n-1})\leq n$ i 
    \footnotesize
    \begin{displaymath}
        f(z)-B_n(z)=\frac{zg(z)+f(0)}{1+\overline{f(0)}zg(z)}-\frac{zB_{n-1}(z)+f(0)}{1+\overline{f(0)}zB_{n-1}(z)}=\frac{(1-\abs{f(0)}^2)z(g(z)-B_{n-1}(z))}{(1+\overline{f(0)}zg(z))(1+\overline{f(0)}zB_{n-1}(z))}.
    \end{displaymath}
    \normalsize
    Per tant $f-B_n$ té un zero d'ordre $n$ en $z=0$ i llavors \begin{displaymath}
        B_n(z)=c_0+c_1z+\dots+c_nz^n+d_{n+1}z^{n+1}+\dots
    \end{displaymath}

    Finalment, és suficient veure que per a tota parcial de $B_{n}$ que convergeix uniformement sobre els compactes de $\D$ a una funció $g$, aquesta funció $g$ és exactament $f$. Llavors, en particular, $B_{n}(z)$ convergirà puntualment a $f(z)$ per a tot $z\in\D$.
    Efectivament, si $B_{n_k}$ tendeix cap a $g$ uniformement sobre compactes, llavors les derivades de $B_{n_k}$ també. Fixant $1\leq j\leq n_k$, 
    \begin{displaymath}
        B_{n_k}^{(j)}(0) \xrightarrow[n_k\to\infty]{} g^{(j)}(0),
    \end{displaymath}
    però com $j\leq n_k$, llavors $B_{n_k}^{(j)}(0)=f^{(j)}(0)$ i per tant $g^{(j)}(0)=f^{(j)}(0)$. Pel Principi de Prolongació Analítica, 
    \begin{displaymath}
        g\equiv f.
    \end{displaymath}

\end{proof}
Per últim, un lema que ens ajudarà amb la demostració del Teorema de Pick.

\begin{lemma}
    Siguin $z_1, z_2\in\D$ diferents i siguin $ w _1, w _2\in\C$. Les següents afirmacions són equivalents:
    \begin{enumerate}[(i)]
        \item Existeix $f$ un $\D$-holomorfisme tal que $f(z_1)=w_1$, $f(z_2)=w_2$.
        \item La forma quadràtica \(\displaystyle Q_2(t_1,t_2)=\sum_{j,k=1}^2\frac{1-w_j\overline{w_k}}{1-z_j\overline{z_k}}t_j\overline{t_k}\geq0\).
        \item \(\displaystyle \abs{\frac{w_2-w_1}{1-\overline{w_1}w_2}}\leq\abs{\frac{z_2-z_1}{1-\overline{z_1}z_2}}\).
        \item \(\displaystyle \frac{(1-\abs{w_2}^2)(1-\abs{w_1}^2)}{\abs{1-\overline{w_1}w_2}^2}\geq\frac{(1-\abs{z_1}^2)(1-\abs{z_2}^2)}{\abs{1-\overline{z_1}z_2}^2}\).
    \end{enumerate}
\end{lemma}

\begin{proof} Veurem $(i)\iff(iii)\iff(iv)\iff(ii)$.

    \large
    $(i)\implies(iii)$
    \normalsize
    És el Lema de Schwarz-Pick \ref{lema:SP}.
    
    \large
    $(iii)\implies(i)$
    \normalsize
    Construïm $f=\tau_3^{-1}\circ\tau_2\circ\tau_1$ a partir de la figura (\ref{fig:lema-prev-pick}).
    \begin{displaymath}
        \tau_1(z)=\frac{z-z_1}{1-\overline{z_1}z},\quad \tau_2(z)=z\frac{\frac{w_2-w_1}{1-\overline{w_1}w_2}}{\frac{z_2-z_1}{1-\overline{z_1}z_2}},\quad \tau_3(w)=\frac{w-w_1}{1-\overline{w_1}w}.
    \end{displaymath}
    $\tau_1$ i $\tau_3$ són automorfismes del disc. Com $(iii)$ es compleix, $t_2$ porta $\D$ en $\D$ i per tant $f$ és el $\D$-holomorfisme de $(i)$.

    \begin{figure}[H]
        \centering
        \usetikzlibrary{arrows}
        \usetikzlibrary{calc}
        \begin{tikzpicture}[scale = 1.4,>=stealth']
            \draw (0,0) circle [radius=1cm];
            \draw[->] (-1.25,0) -- (1.25,0) coordinate (x axis);
            \draw[->] (0, -1.25) -- (0, 1.25) coordinate (y axis);
            \draw (1 cm,1pt) -- (1 cm,-1pt) node[anchor=north,fill=white] {1};
            \coordinate[label=right:$z_1$] (B) at (0.5, -0.3);
            \coordinate[label=right:$z_2$] (C) at (-0.4, 0.2);
            \node[right] (A) at (1.3,0) {\space};
            \node[left] (H) at (0,-1) {\space};
            \fill[black] (B) circle (1pt);
            \fill[black] (C) circle (1pt);
        
            \draw (5,0) circle [radius=1cm];
            \draw[->] (3.75,0) -- (6.25,0) coordinate (x axis);
            \draw[->] (5, -1.25) -- (5, 1.25) coordinate (y axis);
            \draw (6 cm,1pt) -- (6 cm,-1pt) node[anchor=north,fill=white] {1};
            \coordinate[label=left:{\tiny$\displaystyle\frac{z_2-z_1}{1-\overline{z_1}z_2}$}] (D) at (4.8, 0.2);
            \coordinate (E) at (5, 0);
            \node[left] (G) at (3.7,0) {\space};
            \node[right] (R) at (5,-1) {\space};
            \fill[black] (D) circle (1pt);
            \fill[black] (E) circle (1pt) node[anchor=north, fill=white] {$0$};
        
        
            \draw (0,-4) circle [radius=1cm];
            \draw[->] (-1.25,-4) -- (1.25,-4) coordinate (x axis);
            \draw[->] (0, -5.25) -- (0, -2.75) coordinate (y axis);
            \coordinate (aux1) at (1, -4);
            \draw ($(aux1) + (0, 1 pt)$) -- ($(aux1) + (0,-1pt)$) node[anchor=north,fill=white] {1};
            \coordinate[label=below:$w_1$] (I) at (-.5, -4);
            \coordinate[label=right:$w_2$] (J) at (0.3, -4.6);
            \node[right, fill=white] (K) at (1.3,-4) {\space};
            \node[left] (L) at (0,-3) {\space};
            \fill[black] (I) circle (1pt);
            \fill[black] (J) circle (1pt);
        
        
            \draw (5,-4) circle [radius=1cm];
            \draw[->] (3.75,-4) -- (6.25,-4) coordinate (x axis);
            \draw[->] (5, -5.25) -- (5, -2.75) coordinate (y axis);
            \coordinate (aux2) at (6, -4);
            \draw ($(aux2) + (0, 1 pt)$) -- ($(aux2) + (0,-1pt)$) node[anchor=north,fill=white] {1};
            \coordinate[label=left:{\tiny$\displaystyle\frac{w_2-w_1}{1-\overline{w_1}w_2} $}] (M) at (5.5, -3.3);
            \coordinate (N)  at (5, -4);
            \node[left] (P) at (3.7,-4) {\space};
            \node[right] (Q) at (5,-3) {\space};
            \fill[black] (M) circle (1pt);
            \fill[black] (N) circle (1pt) node[anchor=north, fill=white] {$0$};
        
        
            \path[->]
            (A) edge node [pos=.5, above] {$\displaystyle \tau_1=\frac{z-z_1}{1-\overline{z_1}z}$} node[pos=.5, below] {\tiny$\displaystyle\begin{aligned} z_1&\mapsto 0 \\
                z_2 &\mapsto  \frac{z_2-z_1}{1-\overline{z_1}z_2} \end{aligned}$ } (G)
            (H) edge[bend right, dashed] node [left] {$f$} (L)
            (K) edge node [pos=.5, above] {$\displaystyle \tau_3=\frac{w-w_1}{1-\overline{w_1}w}$} node[pos=.5, below] {\tiny$\displaystyle\begin{aligned} w_1&\mapsto 0 \\
                w_2 &\mapsto  \frac{w_2-w_1}{1-\overline{w_1}w_2} \end{aligned}$ } (P)
            (R) edge[bend left] node [right] {$\displaystyle \tau _{2} =z\frac{\frac{w_{2} -w_{1}}{1-\overline{w_{1}} w_{2}}}{\frac{z_{2} -z_{1}}{1-\overline{z_{1}} z_{2}}}$} node[pos=0.3, below, xshift=-1.2cm] {\tiny$\displaystyle\begin{aligned} 0&\mapsto 0 \\
                 \tau_1(z_2) &\mapsto  \frac{w_2-w_1}{1-\overline{w_1}w_2}\end{aligned}$ }(Q);
        \end{tikzpicture}

        \caption{$f=\tau_3^{-1}\circ\tau_2\circ\tau_1$}\label{fig:lema-prev-pick}
    \end{figure}

    \large
    $(iii)\iff(iv)$
    \normalsize
    És un càlcul. Efectivament,
    \begin{align*}
        &\frac{\abs{w_2-w_1}}{\abs{1-\overline{w_1}w_2}}\leq\frac{\abs{z_2-z_1}}{\abs{1-z_2\overline{z_1}}} \iff 1-\abs{\frac{w_2-w_1}{1-\overline{w_1}w_2}}^2\geq1-\abs{\frac{z_2-z_1}{1-\overline{z_1}z_2}}^2\\
        &1-\abs{\frac{w_2-w_1}{1-\overline{w_1}w_2}}^2=\frac{\abs{1-\overline{w_1}w_2}^2-\abs{w_1-w_2}^2}{\abs{1-\overline{w_1}w_2}^2}=\frac{1-\abs{w_1}^2-\abs{w_2}^2+\abs{w_1w_2}^2}{\abs{1-\overline{w_1}w_2}^2}.
    \end{align*}
    D'altra banda,
    \begin{displaymath}
        \frac{(1-\abs{w_1}^2)(1-\abs{w_2}^2)}{\abs{1-\overline{w_1}w_2}^2}=\frac{1-\abs{w_1}^2-\abs{w_2}^2+\abs{w_1w_2}^2}{\abs{1-\overline{w_1}w_2}^2}.
    \end{displaymath}
    Pel mateix càlcul ho tenim per \(\displaystyle \abs{\frac{z_2-z_1}{1-\overline{z_1}z_2}}\) , per tant $(iii)\iff(iv)$.   

    \large
    $(ii)\iff(iv)$
    \normalsize
    \begin{align*}
        (ii) &= \sum_{j,k=1}^{2}\frac{1-w_j\overline{w_k}}{1-z_j\overline{z_k}}t_j\overline{t_k}\geq0\iff  
        \begin{pmatrix}
            \frac{1-\left | w_1 \right |^2}{1-\left | z_1 \right |^2}& \frac{1-w_1\overline{w_2}}{1-z_1\overline{z_2}}\\ 
            \frac{1-w_2\overline{w_1}}{1-z_2\overline{z_1}}& \frac{1-\left | w_2 \right |^2}{1-\left | z_2 \right |^2}
        \end{pmatrix}\, \text{és definida positiva.}\\
        &\iff 
        \begin{cases}
            \frac{1-\left | w_1 \right |^2}{1-\left | z_1 \right |^2}\geq0\, \text{(cert).}\\
            \text{i}\\
            \frac{1-\left | w_1 \right |^2}{1-\left | z_1 \right |^2}\frac{1-\left | w_2 \right |^2}{1-\left | z_2 \right |^2}-\abs{\frac{1-w_2\overline{w_1}}{1-z_2\overline{z_1}}}^2\geq0.
        \end{cases}
    \end{align*}
    Aquesta última desigualtat ens porta a 
    \begin{align*}
        &\frac{1-\left | w_1 \right |^2}{1-\left | z_1 \right |^2}\frac{1-\left | w_2 \right |^2}{1-\left | z_2 \right |^2}-\abs{\frac{1-w_2\overline{w_1}}{1-z_2\overline{z_1}}}^2\geq0\\
        &\iff\frac{(1-\abs{w_1}^2)(1-\abs{w_2}^2)}{(1-\abs{z_1}^2)(1-\abs{z_2}^2)}\geq\frac{\abs{1- \overline{w_1}w_2}^2}{\abs{1-\overline{z_1}z_2}^2}\\
        &\iff\frac{(1-\abs{w_1}^2)(1-\abs{w_2}^2)}{\abs{1- \overline{w_1}w_2}^2}\geq\frac{(1-\abs{z_1}^2)(1-\abs{z_2}^2)}{\abs{1-\overline{z_1}z_2}^2},
    \end{align*}
    que és la condició $(iv)$. Això acaba la demostració del lema.
\end{proof}

\subsection{El Problema d'interpolació}
Per fi sóm en condicions de resoldre el problema que dóna nom a aquest treball.

Siguin $\{z_1,\dots,z_n\}$ un conjunt finit de punts diferents de $\D$. Pick va determinar  $\{w_1,\dots,w_n\}$ per als quals el problema d'interpolació

\begin{equation}\label{eq:interpol}
    f(z_j)=w_j,\, \text{  } j=1,2,\dots,n
\end{equation}
%
té una solució $\D$-holomorfa $f(z)$.

\begin{theorem}[Pick]\label{th:pick}
    Existeix $f$ $\D$-holomorfisme que satisfà la interpolació \normalfont{(\ref{eq:interpol})} si i només si la forma quadràtica 
    \begin{displaymath}
        Q_n(t_1,\dots,t_n)=\sum_{j,k=1}^n=\frac{1-w_j\overline{w_k}}{1-z_j\overline{z_k}}t_j\overline{t_k}\geq0
    \end{displaymath}
    per a tot $t_1,\dots, t_n\in\mathbb{R}$. En aquest cas, existeix un producte de Blaschke finit de grau com a molt $n$ que resol \normalfont{(\ref{eq:interpol})}.
\end{theorem}

\begin{proof}
    Farem inducció sobre $n$. Si $n=1$, cal veure que existeix $f$ $\D$-holomorfa tal que 
    \begin{displaymath}
        f(z_1)=w_1\iff\frac{1-\abs{w_1}^2}{1-\abs{z_1}^2}\abs{t_1}^2\geq0.  
    \end{displaymath}
    
    Com $1-\abs{z_1}^2\geq0$, és equivalent dir que $\abs{w_1}\leq1$. Per tant s'ha de veure que existeix $f$ $\D$-holomorfisme amb $f(z_1)=w_1$ si i només si $\abs{w_1}\leq1$. La necessitat és evident, ja que $\norm{f}_\infty\leq1$. Recíprocament, si $\abs{w_1}\leq1$, podem triar $f$ la funció constant igual a $w_1$.
    
   
    Vist el cas $n=1$, suposem que $n > 1$. Suposem que existeix la funció $f$ que resol el problema d'interpolació. Com és $\D$-holomorfa, llavors $\abs{w_n}\leq1$.
    Si $\abs{w_n}=1$, llavors pel principi del mòdul màxim, $f\equiv w_n$ i $w_j=w_n$ per a tot $j$.

    Suposem que $Q_n\geq0$. Si fixem $t_n=1$, $t_j=0$  $j<n$ Tenim
    \begin{displaymath}
        0\leq\sum_{j,k=1}^{n}\frac{1-w_j\overline{w_k}}{1-z_j\overline{z_k}}t_j\overline{t_k}=\frac{1-\abs{w_n}^2}{1-\abs{z_n}^2}\implies\abs{w_n}^2\leq1.
    \end{displaymath}
    
    Si $\abs{w_n}=1$, llavors fixem un $k_0$ i $t_j=0$ $\forall j\neq k_0,n$. Llavors 
    
    \noindent $Q_n(0,\dots,0,t_{k_0},0,\dots,0,t_n)$ és equivalent a $Q_2(t_{k_0},t_n)$. Pel lema anterior tenim 
    \begin{displaymath}
        \abs{\frac{w_n-w_{k_0}}{1-\overline{w_{k_0}}w_n}}\leq\abs{\frac{z_n-z_{k_0}}{1-\overline{z_{k_0}}z_n}}.
    \end{displaymath}
    Com $\frac{z-z_{k_0}}{1-\overline{z_{k_0}}z}$ és automorfisme del disc i $\abs{z_n}<1$, tenim 
    \begin{displaymath}
        \abs{\frac{z_k-z_{k_0}}{1-\overline{z_{k_0}}z_n}}<1.
    \end{displaymath}
    D'altra banda, $\frac{z-w_{k_0}}{1-\overline{w_{k_0}}z}$ és un producte de Blaschke per tant té mòdul 1 a $\partial\D$. Com $\abs{w_n}=1$, 
    \begin{displaymath}
        \abs{\frac{w_n-w_{k_0}}{1-\overline{w_{k_0}}w_n}}=1.
    \end{displaymath}
    Llavors 
    \begin{displaymath}
        1=\abs{\frac{w_n-w_{k_0}}{1-\overline{w_{k_0}}w_n}}<1.
    \end{displaymath}
    Aquest quocient no té sentit i l'única opció és $w_n=w_{k_0}$. Com aquest argument es pot aplicar a qualsevol $1\leq k_0\leq n$, tenim que $w_j=w_n$ $\forall j$. Així doncs, si $\abs{w_n}=1$, triem $f\equiv w_n$. $f$ és producte de Blaschke.
    
    Queda clar que sempre $\abs{w_j}\leq1$ i que el cas amb igualtat és trivial, per tant podem assumir que $\abs{w_n}<1$.

    Volem reduir-nos a $n-1$ per tal d'aplicar les hipòtesis d'inducció. Per a això, movem $z_n$ i $w_n$ al 0 amb automorfismes del disc (\ref{fig:pick}).

    \begin{figure}[H]
        \centering
        \begin{tikzpicture}[scale = 1.5,>=stealth']
            \draw (0,0) circle [radius=1cm];
            \draw[->] (-1.25,0) -- (1.25,0) coordinate (x axis);
            \draw[->] (0, -1.25) -- (0, 1.25) coordinate (y axis);
            \draw (1 cm,1pt) -- (1 cm,-1pt) node[anchor=north,fill=white] {1};
            \coordinate[label=right:$z_1$] (B) at (0.5, -0.3);
            \coordinate[label=right:$z_n$] (C) at (-0.4, 0.2);
            \node[right] (A) at (1.3,0) {\space};
            \node[left] (H) at (0,-1) {\space};
            \fill[black] (B) circle (1pt);
            \fill[black] (C) circle (1pt);

            \draw (5,0) circle [radius=1cm];
            \draw[->] (3.75,0) -- (6.25,0) coordinate (x axis);
            \draw[->] (5, -1.25) -- (5, 1.25) coordinate (y axis);
            \draw (6 cm,1pt) -- (6 cm,-1pt) node[anchor=north,fill=white] {1};
            \coordinate[label=left:$z'_1$] (D) at (4.8, 0.2);
            \coordinate[label=below:{$z'_n=0$}] (E) at (5, 0);
            \node[left] (G) at (3.7,0) {\space};
            \node[right] (R) at (5,-1) {\space};
            \fill[black] (D) circle (1pt);
            \fill[black] (E) circle (1pt);


            \draw (0,-3) circle [radius=1cm];
            \draw[->] (-1.25,-3) -- (1.25,-3) coordinate (x axis);
            \draw[->] (0, -4.25) -- (0, -1.75) coordinate (y axis);
            \coordinate (aux1) at (1, -3);
            \draw ($(aux1) + (0, 1 pt)$) -- ($(aux1) + (0,-1pt)$) node[anchor=north,fill=white] {1};
            \coordinate[label=below:$w_1$] (I) at (-.5, -3);
            \coordinate[label=right:$w_n$] (J) at (0.3, -3.6);
            \node[right, fill=white] (K) at (1.3,-3) {\space};
            \node[left] (L) at (0,-2) {\space};
            \fill[black] (I) circle (1pt);
            \fill[black] (J) circle (1pt);


            \draw (5,-3) circle [radius=1cm];
            \draw[->] (3.75,-3) -- (6.25,-3) coordinate (x axis);
            \draw[->] (5, -4.25) -- (5, -1.75) coordinate (y axis);
            \coordinate (aux2) at (6, -3);
            \draw ($(aux2) + (0, 1 pt)$) -- ($(aux2) + (0,-1pt)$) node[anchor=north,fill=white] {1};
            \coordinate[label=left:$w'_1$] (M) at (5.5, -2.3);
            \coordinate[label=below:{$w'_n=0$}] (N) at (5, -3);
            \node[left] (P) at (3.7,-3) {\space};
            \node[right] (Q) at (5,-2) {\space};
            \fill[black] (M) circle (1pt);
            \fill[black] (N) circle (1pt);

        
            \path[->]
            (A) edge node [pos=.5, above] {$\displaystyle z'_j=\frac{z_j-z_n}{1-\overline{z_n}z_j}$} (G)
            (H) edge[bend right] node [left] {$f$} (L)
            (K) edge node [pos=.5, above] {$\displaystyle w'_j=\frac{w_j-w_n}{1-\overline{w_n}w_j}$} (P)
            (R) edge[bend left] node [right] {$g$} (Q);
        \end{tikzpicture}
        \caption{}\label{fig:pick}
    \end{figure}
     Sigui
    \begin{displaymath}
        z_j'=\frac{z_j-z_n}{1-\overline{z_n}z_j},\; 1\leq j\leq n;\quad w_j'=\frac{w_j-w_n}{1-\overline{w_n}w_j},\; 1\leq j\leq n.
    \end{displaymath}
    Definim 
    \begin{displaymath}
        g:=\frac{f\left( \frac{z+z_n}{1-\overline{z_n}z} \right)-w_n}{1-\overline{w_n}f\left( \frac{z+z_n}{1-\overline{z_n}z} \right)}.
    \end{displaymath}
    Llavors existeix $f$ $\D$-holomorfa que resol (\ref{eq:interpol}) si i nomès si $g$ és $\D$-holomorfa i resol 
    
    \begin{equation}
        g(z_j')=w_j',\quad 1\leq j\leq n.
    \end{equation} 

    A més, $f$ és un producte de Blaschke de grau com a molt $n$ si i només si $g$ també ho és.

    D'altra banda, la forma quadràtica $Q'_n$ corresponent als punts $\{z'_1,...,z_n'\}$ i
    
    \noindent $\{w'_1,...,w'_n\}$ té una forta relació amb $Q_n$.
    Definim 
    \begin{displaymath}
        \alpha_j=\frac{(1-\abs{z_n}^2)^{1/2}}{1-\overline{z_n}z_j},\quad \beta_j=\frac{(1-\abs{w_n}^2)^{1/2}}{1-\overline{w_n}w_j}.
    \end{displaymath}
    Llavors 
    
    \begin{equation}\label{eq:zprim}
        \frac{1-z'_j\overline{z'_k}}{1-z_j\overline{z_k}}=\frac{1-\frac{z_j-z_n}{1-\overline{z_n}z_j}\overline{\left(\frac{z_k-z_n}{1-\overline{z_n}z_k}\right)}}{1-z_j\overline{z_k}}=\frac{(1-\overline{z_n}z_j)(1-z_n\overline{z_k})-(z_j-z_n)(\overline{z_k}-\overline{z_n})}{(1-z_j\overline{z_k})(1-\overline{z_n}z_j)(1-z_n\overline{z_k})}.
    \end{equation}

Veiem que \(\displaystyle \alpha_j\overline{\alpha_k}=\frac{1-\abs{z_n}^2}{(1-z_j\overline{z_n})(1-\overline{z_k}z_n)}\) és igual a (\ref{eq:zprim}):

Efectivament, 
\begin{align*}
    &\frac{(1-\overline{z_n}z_j)(1-z_n\overline{z_k})-(z_j-z_n)(\overline{z_k}-\overline{z_n})}{(1-z_j\overline{z_k})\cancel{(1-\overline{z_n}z_j)}\cancel{(1-z_n\overline{z_k})}}=\frac{1-\abs{z_n}^2}{\cancel{(1-\overline{z_n}z_j)}\cancel{(1-z_n\overline{z_k})}}\\
    \implies &\frac{1\cancel{-\overline{z_n}z_j}\cancel{-z_n\overline{z_k}}+\abs{z_n}^2z_j\overline{z_k}\cancel{+z_j\overline{z_n}}-z_j\overline{z_k}\cancel{+\overline{z_k}z_n}-\abs{z_n}^2}{(1-z_j\overline{z_k})}=1-\abs{z_n}^2\\
    \implies &\frac{1-\abs{z_n}^2+z_j\overline{z_k}\abs{z_n}^2-z_j\overline{z_k}}{(1-z_j\overline{z_k})}=\frac{(1-\abs{z_n}^2)\cancel{(1-z_j\overline{z_k})}}{\cancel{(1-z_j\overline{z_k})}}=1-\abs{z_n}^2.
\end{align*}
Pel mateix càlcul tenim \(\displaystyle \beta_j\overline{\beta_k}=\frac{1-\abs{w_n}^2}{(1-\overline{w_n}w_j)(1-\overline{w_k})}\).

Per tant, 
\begin{displaymath}
    \frac{1-w_j'\overline{w_k'}}{1-z_j'\overline{z_k'}}t_j\overline{t_k}=\frac{1-w_j\overline{w_k}}{1-z_j\overline{z_k}}\left( \frac{\beta_j}{\alpha_j} \right)t_j\overline{\left( \frac{\beta_k}{\alpha_k} \right)t_k}
\end{displaymath}
i 

\begin{equation}
    Q'_n(t_1,\dots,t_n)=Q_n\left(\frac{\beta_1}{\alpha_1}t_1,\dots,\frac{\beta_n}{\alpha_n}t_n\right).
\end{equation}

Així doncs, $Q'_n\geq0 \iff Q_n\geq0$ i hem reduït el problema al cas $z_n=w_n=0$.

Suposem, per tant, $z_n=w_n=0$. Existeix una $f$ $\D$-holomorfa tal que $f(0)=0$ i 
\begin{displaymath}
    f(z_j)=w_j,\quad 1\leq j \leq n-1
\end{displaymath}
si i només si existeix $g(z)=\frac{f(z)}{z}$ un $\D$-holomorfisme tal que 

\begin{equation}\label{eq:interpol_redux}
    g(z_j)=\frac{w_j}{z_j},\quad 1\leq j\leq n-1.
\end{equation}

A més, $f$ és producte de Blaschke de grau $d$ si i només si $g$ ho és de grau $d-1$.

Per la hipòtesi d'inducció, (\ref{eq:interpol_redux}) té solució si i només si 
\begin{displaymath}
    \tilde{Q}_{n-1}(s_1,\dots,s_{n-1})=\sum_{j,k=1}^{n-1}\frac{1-\left( \frac{w_j}{z_j} \right)\overline{\left( \frac{w_k}{z_k} \right)}}{1-z_j\overline{z_k}}s_j\overline{s_k}\geq0,
\end{displaymath}
per a tot $s_1,\dots,s_{n-1}\in\mathbb{R}$. El teorema es redueix a veure que 
\begin{displaymath}
    Q_n\geq0\iff\tilde{Q}_{n-1}\geq0
\end{displaymath}
sota la condició $w_n=z_n=0$.

Com $z_n=w_n=0$, 
\begin{displaymath}
    Q_n(t_1,\dots,t_n)=\abs{t_n}^2+2\text{Re}{\sum_{j=1}^{n-1}\overline{t_j}t_n}+\sum_{j,k=1}^{n-1}\frac{1-w_j\overline{w_k}}{1-z_j\overline{z_k}}t_j\overline{t_k}.
\end{displaymath}
Si completem quadrats per a $t_n$, tenim 
\begin{displaymath}
    Q_n(t_1,\dots,t_n)=\abs{t_n+\sum_{j=1}^{n-1}t_j}^2+\sum_{j,k=1}^{n-1}\left( \frac{1-w_j\overline{w_k}}{1-z_j\overline{z_k}}-1 \right)t_j\overline{t_k}.
\end{displaymath}
Ara bé, 
\begin{displaymath}
    \frac{1-w_j\overline{w_k}}{1-z_j\overline{z_k}}-1=\frac{z_j\overline{z_k}-w_j\overline{w_k}}{1-z_j\overline{z_k}}=\frac{1-\left( \frac{w_j}{z_j} \right)\overline{\left( \frac{w_k}{z_k} \right)}}{1-z_j\overline{z_k}}z_j\overline{z_k}
\end{displaymath}
i per tant 
\begin{displaymath}
    Q_n(t_1,\dots,t_n)=\abs{\sum_{j=1}^{n}t_j}^2+\tilde{Q}_{n-1}(z_1t_1,\dots,z_{n-1}t_{n-1}).
\end{displaymath}
Llavors $\tilde{Q}_{n-1}\geq0\implies Q_n\geq0$.

\noindent Finalment, fixant \(\displaystyle t_n=-\sum_{j=1}^{n-1}t_j\) veiem que $Q_n\geq0\implies\tilde{Q}_{n-1}\geq0$.
\end{proof}
\newpage
\section{El treball de Nevanlinna}
No oblidem, però, que el problema d'interpolació (\ref{problema}) porta el nom d'un altre matemàtic, Rolf Nevanlinna.

Nevanlinna, tot i publicar els seu treball més tard, dona uns resultats molt rellevants. Amb una construcció basada en l'\textit{algorisme de Schur}, construeix les solucions del problema i, amb unes hipòtesis extra, construeix també les solucions per la problema amb un nombre infinit de punts. Dit d'una altra manera, amb una condició addicional, Nevanlinna descriu totes les funcions $f$ que són $\D$-holomorfes i tals que 
\begin{equation}\label{eq:problemainf}
    f(z_j)=w_j,\quad j=1,2,\dots.
\end{equation}

Abans d'enunciar aquest teorema, introduïm una mica de notació. Al conjunt del tots els $\D$-holomorfismes el denotarem per $\B$, i al conjunt de les solucions del problema de Pick-Nevanlinna per $n$ punts el denotarem per $E_n$. 
\begin{align*}
    \B &=\left\{ f\colon f \text{ és un } \D\text{-holomorfisme} \right\}\\
    E_n &=\left\{ f\in\B\colon f(z_j)=w_j,\, 1\leq j\leq n \right\}
\end{align*}
Seguint aquesta notació, les solucions de (\ref{eq:problemainf}) formen el conjunt $\Einf$.
Aquesta secció requerirà d'una lleugera noció d'anàlisi funcional i teoria de la mesura. Parlarem de \textbf{gairebé en tot punt} per dir que una propietat és compleix tret d'un conjunt de mesura nul·la.

Per últim, introduïm un concepte nou. Diem que una funció $f\in\B$ és \textbf{inner function} si $\abs{f(e^{i\theta})}=1$ gairebé per a tot $\theta\in\mathbb{R}$.

\begin{theorem}[Nevanlinna]
    Si hi ha dues funcions de \B que satisfan la interpolació \textup{(\ref{eq:problemainf})}, llavors hi ha una inner function que també satisfà \textup{(\ref{eq:problemainf})}.
\end{theorem}

La demostració d'aquest teorema serà orgànica, veient els resultats auxiliars conforme els anem necessitant. Començarem parametritzant $E_n$ (i per tant resolent el problema per a $n$ punts) i posteriorment farem el pas a $\displaystyle\Einf=\cap_n E_n$ per veure finalment que aquest conjunt conté inner functions.

Comencem per $E_1$. Aprofitem la idea que va sortir a la demostració del Teorema de Pick \ref{th:pick}. Per mitjà d'homografies, el problema de trobar $f\in\B$ tal que $f(z_1)=w_1$ es redueix a trobar $h\in\B$ tal que $f(0)=0$.

%%%%%%%%%%%%%%%%%%%%%%%%%%%%%%%%%%%%%%%%%%%%%%%%%%%
%FIGURA
%%%%%%%%%%%%%%%%%%%%%%%%%%%%%%%%%%%%%%%%%%%%%%%%%%%

Llavors $h(z)=zg_1(z)$ per a algun $g_1\in\B$.
\begin{displaymath}
    f=\tau_{w_1}^{-1}\circ h\circ \tau_{z_1}\iff\tau_{w_1}\circ f= h\circ\tau_{z_1}
\end{displaymath}
Com $h(z)=zg_1(z)$, tenim que $h\circ\tau_{z_1}=\tau_{z_1}\cdot(g_1\circ\tau_{z_1})$.
Reescrivim $(g_1\circ\tau_{z_1})(z)$ com $(\tau_{-c_1}\circ f_1)(z)$ per a un cert $c_1\in\D$ i $f_1\in\B$, que podríem escriure en termes de $g_1$ i $c_1$ si fos necessari (no ho serà). El $c_1$ ens serà útil triar-lo més endavant. Tenim, doncs, 

\begin{equation}\label{eq:f1}
    \frac{f-w_1}{1-\bar{w}_1f}=\frac{f_1+c_1}{1+\bar{c}_1f_1}\frac{z-z_1}{1-\bar{z}_1z}.
\end{equation}

Siguin 
\begin{align*}
    A_1(z) &= w_1(1-\bar{z}_1z)+c_1(z-z_1),\\
    B_1(z) &= \bar{c}_1w_1(1-\bar{z}_1z)+(z-z_1),\\
    C_1(z) &= (1-\bar{z}_1z)+c_1\bar{w}_1(z-z_1),\\
    D_1(z) &= \bar{c}_1(1-\bar{z}_1z)+\bar{w}_1(z-z_1),
\end{align*}
llavors per a tota $f_1\in\B$, (\ref{eq:f1}) defineix una funció $f\in E_1$,
\begin{equation}\label{eq:f}
    f(z)=\frac{A_1(z)+B_1(z)f_1(z)}{C_1(z)+D_1(z)f_1(z)}
\end{equation}
i aquesta es la parametrització de $E_1$.

Suposem ara $f\in E_n$, $n\geq2$. Llavors (\ref{eq:f1}) determina $f_1(z_j)$ per $2\leq j\leq n$. Resolent (\ref{eq:f1}) obtenim 
\begin{equation}\label{eq:n1}
    f_1(z_j)=w_j^{(1)},\quad 2\leq j\leq n.
\end{equation}
%
És clar que $\abs{w_j^{(1)}}\leq 1$, ja que $f_1\in\B$. A més, el cas $\abs{w_j^{(1)}}=1$ per a algun $j$ no és possible, ja que llavors pel principi del mòdul máxim $f_1(z)\equiv w_j^{(1)}$, $\abs{z}<1$ i per (\ref{eq:f1}) $E_n$ conté només una funció. Tenim doncs
\begin{equation*}
    \abs{w_j^{(1)}}<1, j = 2,3,\dots.
\end{equation*}

Hem aconseguit poder ignorar el valor $f_1(z_1)$, i ara $E_n$ està definit només per les $n-1$ equacions (\ref{eq:n1}), en lloc de les $n$ originals. Repetint el raonament, per a un cert $c_2\in\D$, escrivim

\begin{equation}\label{eq:f2}
    \frac{f_1-w_2^{(1)}}{1-\bar{w}_2^{(1)}f_1}=\frac{f_2+c_2}{1+\bar{c}_2f_2}\frac{z-z_2}{1-\bar{z}_2z}.
\end{equation}
%
Ara $f\in E_2$ si i només si $f_1(z_2)=w_2^{(1)}$, la qual cosa succeeix si i només si (\ref{eq:f2}) es compleix per a alguna $f_2\in\B$. A més, quan $n>2$, $f\in E_n$ si i només si
\begin{equation*}
    f_2(z_j)=w_j^{(2)},\quad 3\leq j\leq n,
\end{equation*}
on $w_j^{(2)}$ queda determinat per (\ref{eq:f2}). També tenim $\abs{w_j^{(2)}}<1$ pel mateix raonament que ens donava $\abs{w_j^{(1)}}<1$.

Continuem per inducció, sempre assumint que $E_n$ conté més d'una funció. Per $k\leq n$, $f\in E_n$ si i només si hi ha $f_0,f_1,\dots, f_k\in \B$ tals que $f_0=f$ i 
\begin{displaymath}
    f_k(z_j)=w_j^{(k)},\quad k+1\leq j\leq n
\end{displaymath}
amb $\abs{w_j^{(k)}}<1$ i $w_j^{(0)}=w_j$ i tals que 
\begin{equation}\label{eq:fk1}
    \frac{f_{k-1}-w_k^{(k-1)}}{1-\bar{w}_k^{(k-1)}f_{k-1}}=\frac{f_k+c_k}{1+\bar{c}_kf_k}\frac{z-z_k}{1-\bar{z}_kz}.
\end{equation}
%
on $c_k\in\D$ queden per determinar. Els valors concrets de $w_j^{(k)}$ no ens són rellevants.

Definim
\begin{equation}\label{eq:greek}
    \begin{split}
        \alpha_k(z) &= w_k^{(k-1)}(1-\bar{z}_kz)+c_k(z-z_k),\\
        \beta_k(z) &= \bar{c}_kw_k^{(k-1)}(1-\bar{z}_kz)+(z-z_k),\\
        \gamma_k(z) &= (1-\bar{z}_kz)+c_k\bar{w}_k^{(k-1)}(z-z_k),\\
        \delta_k(z) &= \bar{c}_k(1-\bar{z}_kz)+\bar{w}_k^{(k-1)}(z-z_k),
    \end{split}
\end{equation}
%
i reescrivim (\ref{eq:fk1}) a l'estil de (\ref{eq:f}),
\begin{equation}\label{eq:fk1bis}
    f_{k-1}(z)=\frac{\alpha_k(z)+\beta_k(z)f_k(z)}{\gamma_k(z)+\delta_k(z)f_k(z)}.
\end{equation}

Per inducció, (\ref{eq:fk1bis}) i (\ref{eq:f}) ens donen
\begin{equation}\label{eq:fbis}
        f(z)=\frac{A_n(z)+B_n(z)f_n(z)}{C_n(z)+D_n(z)f_n(z)}
\end{equation}
%
on 
\begin{equation}\label{eq:An}
    \begin{split}
        A_n(z)=\gamma_n(z)A_{n-1}(z)+\alpha_n(z)B_{n-1}(z),\\
        B_n(z)=\delta_n(z)A_{n-1}(z)+\beta_n(z)B_{n-1}(z),\\
        C_n(z)=\gamma_n(z)C_{n-1}(z)+\alpha_n(z)D_{n-1}(z),\\
        D_n(z)=\delta_n(z)C_{n-1}(z)+\beta_n(z)D_{n-1}(z),
    \end{split}
\end{equation}
%
són polinomis de grau com a molt $n$ en $z$. Queda demostrat
\begin{lemma}\label{lema0}
    Siguin $A_n,B_n,C_n,D_n$ els polinomis definits per (\ref{eq:An}) i (\ref{eq:greek}). Llavors $f(z)\in E_n$ si i només si $f$ compleix (\ref{eq:fbis}) per a algun $f_n\in\B$.
\end{lemma}

És hora de parlar dels $c_k$. Fixem $c_k=\bar{z}_kw_k^{(k-1)}$. Això famosa
\begin{displaymath}
    \delta_k(0)=0,\quad k=1,2,\dots,n.
\end{displaymath}
A més, com $D_1(0)=z_1\bar{w}_1(1-0)+\bar{w}_1(0-z_1)=0$ i $D_k(0)=\delta_k(0)C_{k-1}(0)+\beta_k(0)D_k-1(0)$ tenim
\begin{displaymath}
    D_k(0)=0,\quad k=1,2,\dots,n.
\end{displaymath}
Ens serà important en particular que $\delta_n(0)=D_n(0)=0$.

Fix $n$ i $z\in\D$,
\begin{displaymath}
    \hg(w)=\frac{A_n(z)+B_n(z)w}{C_n(z)+D_n(z)w}
\end{displaymath}
és una homografia. Triant $f_n(w)=w$ a (\ref{eq:fbis}), tenim 
\begin{displaymath}
    \hg(\bar{\D})\subseteq\bar{\D}.
\end{displaymath}
Com $\hg$ és una homografia, $\hg(\partial\D)$ és una recta o una circumferència, i com ha de ser dins $\bar\D$, $\hg(\partial\D)$ és necessàriament una circumferència. Llavors $\hg(\bar\D)$ és un disc dins $\D$, diem-li $\Delta_n(z)$.

Per $\abs{z}<1$ tenim doncs que $\left\{ f(z)\colon f\in E_n \right\}$ és un disc tancat $\Delta_n(z)\subseteq\bar\D$ definit per 
\begin{equation}\label{eq:gencirc}
    \left\{ \frac{A_n(z)+B_n(z)w}{C_n(z)+D_n(z)w}\colon\abs{w}\leq1 \right\}.
\end{equation}
%
Per $\abs{z}=1$ la fórmula té sentit, tot  i que potser algun $f\in E_n$ no és definit a $z$.

Busquem ara el radi i centre de $\Delta_n(z)$. Per trobar el centre raonarem geomètricament. L'antiimatge de $\infty$ per $\hg$ és $-C_n/D_n$. Per simetría, la seva reflexió respecte $\partial\D$ serà el centre de $\Delta_n(z)$. Per tant, $ \hg( -\overline{D_n/C_n})$ és el centre de $\Delta_n(z)$.

Per trobar el radi $\rho_n(z)$, observem que per a qualsevol $\abs{w}=1$, 
\begin{displaymath}
    \rho_n(z)=\abs{\hg\left( -\frac{D_n}{C_n} \right)-\hg(w)}
\end{displaymath}
triem $w=1$ i tenim 
\begin{align*}
    \rho_n(z) &=\abs{\frac{A_n\bar{C}_n-B_n\bar{D}_n}{\abs{C_n}^2-\abs{D_n}^2}-\frac{A_n+B_n}{C_n+D_n}}=\abs{\frac{A_nD_n-B_nC_n}{\abs{C_n}^2-\abs{D_n}^2}}\abs{\frac{\bar{C}_n+\bar{D}_n}{C_n+D_n}}\\
    &=\abs{\frac{A_nD_n-B_nC_n}{\abs{C_n}^2-\abs{D_n}^2}}.
\end{align*}

Observem que $\Delta_n(z)$ degenera en un sol punt si i només si $z=z_j$ per a algun $1\leq j\leq n$, ja que
\begin{equation}\label{eq:delta0}
    \begin{split}
    A_nD_n-B_nC_n &=(\gamma_n\beta_n-\alpha_n\delta_n)(A_{n-1}D_{n-1}-B_{n-1}C_{n-1})\\
    &=\prod_{k=1}^n(1-\abs{z_k}^2\abs{w_k^{(k-1)}}^2)(1-\abs{w_k^{(k-1)}}^2)(z-z_k)(1-\bar{z}_kz).
    \end{split}
\end{equation}

El següent pas serà normaltitzar $A_n,B_n,C_n,D_n$. Per a això ens calen dos lemes:
\begin{lemma}\label{lemma1}
    Si $E_n\neq\varnothing$, llavors quan $\abs{z}=1$, $\Delta_n(z)=\bar{\D}$, $\rho_n(z)=1$ i 
    \begin{align*}
        \abs{B_n(z)} &= \abs{C_n(z)}\\
        \abs{A_n(z)} &= \abs{D_n(z)}\\
        \frac{A_n(z)}{C_n(z)} &= \overline{\left( \frac{D_n(z)}{B_n(z)} \right)}=\lambda_n(z),\quad \text{amb } \abs{\lambda_n(z)}<1.
    \end{align*}
\end{lemma}

\newpage
\bibliography{bibliografia}
\bibliographystyle{plain}
\end{document}